\section{Diskussion}
Zunächst einmal scheint die Vorhersage des quantenmechanischen Modells richtig zu sein, dass die Stromstärke
linear mit der Stromstärke zusammenhängt. Dies wird an der Abbildung \ref{fig:blau} deutlich.\\

\begin{table}[H]
    \centering
    \caption{Abweichungen zwischen den direkt gemessenen Grenzspannungen und den indirekt gemessenen Grenzspannungen.}
    \label{tab:fehler}
    \begin{tblr}{
        colspec = {S S[table-format=1.3]},
        row{1}={guard,mode=math}  , 
        vline{3}={2}{-}{text=\clap{$\pm$}},     
    }
    \toprule
    \text{Wellenlänge /}\unit{\nano\meter}    &   \text{Abweichung} &   \\
    \midrule
    404.7  &   0.007  &  0.027 \\
    435.8  &   0.014  &  0.030 \\
    546.0  &   0.210  &  0.050 \\
    577.0  &   0.110  &  0.040 \\
    \bottomrule
    \end{tblr}
\end{table}

An der Tabelle \ref{tab:fehler} ist zu erkennen, dass für kleinere Wellenlängen die Abweichungen eher klein sind im
Vergleich zu den größeren. Dies liegt wahrscheinlich daran, dass das Ablesen der Stromstärke für diese hohen
Wellenlängen weniger Schwankungen beim Messen unterlag, bezeihungsweise bei den höheren Stromstärken nicht so auffällig
sind.\\
Die mit beiden Grenzspannungen berechneten Planckkonstanten haben jedoch nur noch eine Abweichung von 
$\qty{0.01(0.09)}{}$ , sie weisen also keine großartigen Fehler mehr auf, wie bei den hohen Wellenlängen.
Auch verglichen mit dem Literaturwert von Plancks Konstante $h=\qty{6.62607015e-34}{\joule\second}$, $\qty{0.07(0.05)}{}$ 
für die erste und $\qty{0.06(0.06)}{}$ für die zweite, sind die berechnungen ziemlich genau für die vorher berechneten 
Abweichungen.
\label{sec:Diskussion}
