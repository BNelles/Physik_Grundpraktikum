\section{Diskussion}
\label{sec:Diskussion}
\subsection{Fourier-Synthese}
Hier kann man erkennen, dass im Vergleich der drei Funktionen die Rechteckspannung fehlerbehafteter aussieht, als die Sägezahnspannung.
Dies ist darauf zurückzuführen, dass bei der Sägezahnspannung der Wertebereich von n nicht so weit eingeschränkt ist wie bei der Rechteckspannung.
Daher kann die doppelte Menge an Oberschwingungen verwendet werden.
Die Dreieckspannung konnte hier deutlich genauer konstruiert werden, als die anderen beiden Funktionen.
Das ist damit zu erklären, dass die Gewichtung der Oberschwingungen hier mit $\frac{1}{n^2}$ abfällt und die späteren Oberschwingungen daher nicht so signifikante Änderungen hervorrufen.
Aber noch eine größere Auswirkung sollte das Gibbsche Phänomen haben.
Durch dieses kann auch bei steigendem n die Funktion an einer Unstetigkeit nicht angenähert werden.
Da bei der Dreieckspannung im Gegensatz zu den anderen Beiden eine komplett stetige Funktion ist, sollte diese auch genauer konstruiert werden können.

\subsection{Fourier-Analyse}
Bei der Fourier-Analyse wird anhand der berechneten Abweichungen in Tabelle \ref{tab:tabelle4} deutlich, dass die experimentell bestimmten Werte kaum von den theoretisch berechneten abweichen.
Das könnte daran liegen, dass bei dem Experiment die Funktionen direkt aus einem Signalgenerator kommen.
Außerdem wird die Fourier-Analyse von dem Oszilloskop selbst durchgeführt und daher ist wenig Raum für Fehler.

 