\section{Durchführung}
\label{sec:Durchführung}
\subsection{Rechteckspannung}
Die Rechteckspannung ist hier als ungerade Funktion definiert.
Daher müssen bei der Beschreibung mit einer Fourier-Reihe, die sich aus Formel \ref{eqn:reihe} ergibt, die Koeffizienten $a_n$ nicht berücksichtigt werden.
Für die anderen Koeffizienten gilt Formel \ref{eqn:b_n}.
Durch Einsetzen und Integrieren ergibt sich 
\begin{equation}
    b_n=\frac{4a}{\pi n} \qquad n\in2k+1 \; k\in\mathbb{N}
    \label{eqn:amp_recht}
\end{equation}
für die Amplituden und damit folgt
\begin{equation}
    f(t)=\sum_{n=1}^{\infty}\Bigl( \frac{4a}{\pi n} \symbf{sin}(n t)\Bigr) n\in2k+1 \; k\in\mathbb{N}
    \label{eqn:recht}
\end{equation}
für die Fourier-Reihe der Rechteckspannung.
%%Bilder und Beträge eingügen

\subsection{Dreieckspannung}
Bei der Dreieckspannung wird auch diese so definiert, dass die Funktion ungerade ist.
Die Berechnung erfolgt daher wie oben mit den Formeln \ref{eqn:reihe} und \ref{eqn:b_n}.
Daher sind die Koeffizienten
\begin{equation}
    b_n=(-1)^k\frac{4a}{\pi n^2} \qquad n\in2k+1 \; k\in\mathbb{N}
    \label{eqn:amp_drei}
\end{equation}   
und die Funktion entspricht
\begin{equation}
    f(t)=\sum_{n=1}^{\infty}\Bigl((-1)^k \frac{4a}{ \pi n^2} \symbf{sin}(n t)\Bigr) n\in2k+1 \; k\in\mathbb{N}
    \label{eqn:drei}
\end{equation}.

\subsection{Sägezahnspannung}
Auch bei der Sägezahnspannung kann diese ungerade definiert werden.
Analog zu den Berechnungen oben folgt
\begin{equation}
    b_n=(-1)^{n+1}\frac{2a}{ n} \qquad n\in2k+1 \; k\in\mathbb{N}
    \label{eqn:amp_säge}
\end{equation} 
für die Amplituden und für die Funktion
\begin{equation}
    f(t)=\sum_{n=1}^{\infty}\Bigl( (-1)^{n+1}\frac{2a}{ n} \symbf{sin}(n t)\Bigr) n\in\mathbb{N}
    \label{eqn:drei}
\end{equation}.


