\section{Zielsetzung}
\label{sec:Zielsetzung}
Ziel dieses Versuches war, eine Fourier Synthese von verschiedenen Funktionen erfolgreich durchzuführen.
Zusätzlich soll mithilfe des Fourierschen Theorems verschiedene Funktionen analysiert werden.

\section{Theorie}
\label{sec:Theorie}
Nach dem Theorem von Fourer kann jede periodische Funktion $f(t)$ durch Sinus- und Cosinusfunktionen darstellen kann.
Dies geschieht in der Form 
\begin{equation}
    f(t)=\frac{1}{2}a_0+\sum_{n=1}^{\infinity}\Bigl(a_n \symbf{cos}(n \frac{2 \pi}{T} t) + b_n \symbf{sin}(n \frac{2\pi}{T}t)\Bigr) \qquad n\in\mathbb{N}
    \label{eqn:f_reihe}
\end{equation}, unter der Vorraussetzung das die Reihe gleichmäßig konvergiert.
Dabei entspricht T der Periodendauer der Funktion und die Koeffizienten werden durch
\begin{equation}
    a_n=\frac{2}{T}\int_0^T f(t) \symbf{cos}(n \frac{2\pi}{T}t) \symup{d}t \qquad n\in\mathbb{N}
    \label{eqn:a_n}
\end{equation}
und 
\begin{equation}
    b_n=\frac{2}{T}\int_0^T f(t) \symbf{sin}(n \frac{2\pi}{T}t) \symup{d}t \qquad n\in\mathbb{N}
    \label{eqn:b_n}
\end{equation}

\cite{sample}
