\section{Zielsetzung}
\label{sec:Zielsetzung}
Ziel dieses Versuches war, eine Fourier Synthese von verschiedenen Funktionen erfolgreich durchzuführen.
Zusätzlich soll mithilfe des Fourierschen Theorems verschiedene Funktionen analysiert werden.

\section{Theorie}
\label{sec:Theorie}
\subsection{Fourier-Analyse}
Nach dem Theorem von Fourer kann jede periodische Funktion $f(t)$ durch Sinus- und Cosinusfunktionen darstellen kann.
Dies geschieht in der Form 
\begin{equation}
    f(t)=\frac{1}{2}a_0+\sum_{n=1}^{\infty}\Bigl(a_n \symbf{cos}(n \frac{2 \pi}{T} t) + b_n \symbf{sin}(n \frac{2\pi}{T}t)\Bigr) \qquad n\in\mathbb{N}
    \label{eqn:f_reihe}
\end{equation}, unter der Vorraussetzung das die Reihe gleichmäßig konvergiert.
Dabei entspricht T der Periodendauer der Funktion und die Amplituden werden durch
\begin{equation}
    a_n=\frac{2}{T}\int_0^T f(t) \symbf{cos}(n \frac{2\pi}{T}t) \symup{d}t \qquad n\in\mathbb{N}
    \label{eqn:a_n}
\end{equation}
und 
\begin{equation}
    b_n=\frac{2}{T}\int_0^T f(t) \symbf{sin}(n \frac{2\pi}{T}t) \symup{d}t \qquad n\in\mathbb{N}
    \label{eqn:b_n}
\end{equation}
bestimmt.
Dabei gibt es eine Grundschwingung mit der Grundfrequenz $v_1=\frac{1}{T}$.
Die Schwingungen,die mit der Grundschwingung die Funktion $f(t)$ ergeben, nennt man Oberschwingungen, von welchen die Frequenzen ganzzahlige Vielfache von $v_1$ sind.
Es treten hier nur Phasenverschiebungen von 0, $\frac{1}{2}\pi$, $\pi$ und $\frac{3}{2}\pi$ auf.
Wenn eine Funktion, von der eine Fourier Analyse durchgeführt werden soll, gerade ist, sind alle $b_n=0$.
Bei einer ungeraden Funktion fallen hingegen alle $a_n$ weg.
Wenn die Funktion $f(t)$ nicht stetig ist, wird auch mit steigendem n die Funktion an der Stelle, an der sie nicht stetig ist, nicht genauer approximiert.
Dies nennt man Gibbsches Phänomen.

\subsection{Fourier-Transformation}
Mit
\begin{equation}
    g(v)=\int_{-\infty}^{\infty} f(t) e^{ivt} \symup{d}t
    \label{eqn:trafo}
\end{equation}
kann das Frequenzspektrum $g(v)$ von $f(t)$ berechnet werden.%%Wenden wir die überhaupt an?
\cite{V351}
