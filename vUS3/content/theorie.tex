\section{Theorie}
\label{sec:Theorie}

\subsection{Dopplereffekt}

    Schallwellen bewegen sich durch Materialien als Druckwelle, wobei die Ausbreitungsgeschwindigkeit hier eine
    Materialspezifische konstante c ist. Wenn diese Wellen von ein Medium in ein anderes übergehen wird ein Teil von
    ihnen reflektiert. Der Dopplereffekt tritt nun auf, wenn sich das Objekt bewegt, an dem der Schall reflektiert wird.
    Dabei wird die Frequenz $\nu_0$ größer, wenn sich das Objekt entgegen der Ausbreitungsrichtung der Wellen bewegt 
    (positives $v$) und kleiner, wenn es sich in diese Richtung bewegt (negatives $v$). Die neue Frequenz $\nu_1$ kann 
    entsprechend der Gleichung
    \begin{equation}
        \nu_1=\frac{\nu_0}{1-\frac{v}{c}}
        \label{eqn:doppler}
    \end{equation}
    \noindent berechnet werden. Alternativ wird die Frequenzverschiebung $\increment \nu$ über
    \begin{equation}
        \increment \nu=\nu_0\frac{v}{c}(\symup{cos}(\alpha)+\symup{cos}(\beta))
        \label{eqn:verschiebung}
    \end{equation}
    bestimmt. Hierbei ist $\alpha$ der Winkel zwischen dem Wellenvektor der einlaufenden Welle und der Geschwindigkeit
    des sich bewegenden Objektes und $\beta$ der Winkel zwischen dem Wellenvektor der auslaufenden Welle und dem Objekt.\\
    \noindent Falls $\alpha=\beta$ ist vereinfacht sich die Gleichung \ref{eqn:verschiebung} zu
    \begin{equation}
        \increment \nu=2\nu_0\frac{v}{c}\symup{cos}(\alpha)
        \label{eqn:verschiebung2}
    \end{equation}
    \noindent Bei einem Materialwechsel, also einer Änderung in der Schallgeschwindigkeit von $\symup{c}_1$ zu $\symup{c}_2$
    ist der neue Einfallwinkel relativ zum Einfallswinkel $\theta$
    \begin{equation}
        \alpha=\frac{\pi}{2}-\symup{arcsin}\left(\symup{sin}(\theta)\frac{c_1}{c_2}\right) \text{ .}
        \label{eqn:winkel}
    \end{equation}
    
\subsection{Ultraschall}

    Ultraschallwellen sind Schallwellen, welche sich in einem Frequenzbereich von $\qty{20}{\kilo\hertz} 
    \text{ bis }\qty{1}{\giga\hertz}$ befinden. Das heißt, dass diese für den Menschen nicht mehr wahrnehmbar sind.
    Diese Wellen können jedoch gut zwischen verschiedenen Medien wechseln und finden somit Anwendungen zum Beispiel
    in der Medizin für Ultraschallbilder, wo die Schwingungen durch die Haut durchdringen können.\\
    \noindent Generiert werden solche Ultraschallwellen mithilfe des Piezoelektrischen Effektes. Dabei wird ein
    Kristall in einem elektrischen Feld zum Schwingen angeregt und erzeugt somit Druckwellen. Umgekehrt kann ein solcher
    Kristall auch Schallwellen empfangen, wobei dieser dann durch eintreffende Wellen zu Schwingen angeregt wird.


\cite{VUS3}
