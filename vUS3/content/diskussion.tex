\section{Diskussion}

Die Abweichung der Steigung der Gerade in Abbildung \ref{fig:plot} von seinem theoretischen Wert beträgt
hier $\qty{9(11)}{\percent}$. Daraus lässt sich schließen, dass die Geschwindigkeit durch Gleichung \ref{eqn:verschiebung2}
relativ gut bestimmt ist.\\
\noindent Desweiteren zeigen sich die Geschwindigkeitsverläufe für die unterschiedlichen Messtiefen zunächst wie 
erwartet, in der Hinsicht, dass die Flüssigkeit am Rand am langsamsten fließt. Dies geschieht durch die 
Haftreibung mit den Wänden, die zur Mitte hin abnimmt. Es ist jedoch unklar, warum bei drei der Messungen die 
Stromgeschwindigkeit wieder zunimmt, obwohl der Durchmesser des Rohres überschritten wurde.

\label{sec:Diskussion}
