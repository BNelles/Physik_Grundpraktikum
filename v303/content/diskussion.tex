\section{Diskussion}

Anhand der Abbildungen \ref{fig:phi0} bis \ref{fig:phi360} kann eine Phasenabhängigkeit von 360° erkannt werden.
Dies ist auch für \ref{fig:n_phi0} bis \ref{fig:n_phi360} der Fall.
Die Abweichung $A_{N,e}$,also die auf dei Ausgangspannung wirkende Störung durch den Noise Generator, ist mit maximal $\qty{5}{\percent}$ relativ gering, obwohl die eingestellte Störung für $U_{sig}$ ungefähr der Größenordnung von $U_{sig}$ entsprach.
Damit ist die Funktionsweise des Lock-In-Verstärkers, der stark verrauschte Signale reinigen soll, gezeigt. 
In der Tabelle \ref{tab:tabelle1} sind Abweichungen
von bis zu $\qty{50.31}{\percent}$ möglich.
Auch wenn die Werte nicht den erwarteten entspechen, ist die Form des Verlaufes nach \ref{fig:plot} cosinusförmig, was nach \ref{eqn:cos} zu erwarten ist.
Damit ist der Verlauf von $U_{out}$ nur von der Form her der erwartete.
Die großen Abweichungen stammen potentiell von Fehlfunktionen des Gerätes, mit dem es während der Durchführung teilweise Probleme gab.\\

\noindent 
Die maximale Entfernung, bei der noch ein Signal gemessen werden konnte, beträgt $r_{max}=\qty{127}{\centi\meter}$.
Die Spannung $U_{p0}$ könnte aufgrund der konstanten Beleuchtung des Raumes vorhanden sein.
Bei $r_{max}$ ist das Rauschen druch die Raumbeleuchtung und die Streuung der Photodiode so hoch, dass der Lock-In-Verstärker diese nicht mehr ausgleichen kann.

\label{sec:Diskussion}
