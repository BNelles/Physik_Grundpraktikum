\section{Auswertung}
\label{sec:Auswertung}


\begin{table}
  \centering
  \caption{Messreihe bei einem Abstand von $\qty{6}{\centi\meter}$.}
  \label{tab:tabelle}
  \sisetup{table-format=1.1, per-mode=reciprocal}
  \begin{tblr}{
      colspec = {S S S},
      row{1} = {guard, mode=math},
    }
    \toprule
    p \mathbin{/} \unit{\milli\bar} & channel &  Pulse & \\
    \midrule
    0     &  871  &   19898 \\
    50    &  791  &   20435 \\
    100   &  808  &   17321 \\
    150   &  787  &   17543 \\
    200   &  760  &   17198 \\
    250   &  702  &   16791 \\
    300   &  718  &   15904 \\
    350   &  624  &   15343 \\
    400   &  732  &   1604  \\
    450   &  724  &   21    \\
    500   &  726  &   1     \\
    550   &       &   0     \\
    \bottomrule
  \end{tblr}
\end{table}

\begin{table}
  \centering
  \caption{Messreihe bei einem Abstand von $\qty{5}{\centi\meter}$.}
  \label{tab:tabelle}
  \sisetup{table-format=1.1, per-mode=reciprocal}
  \begin{tblr}{
    colspec = {S S S},
    row{1} = {guard, mode=math},
    }
    \toprule
    p \mathbin{/} \unit{\milli\bar} & channel &  Pulse & \\
    \midrule
    0     &  1023   &  23373  \\
    50    &  919    &  22738  \\
    100   &  875    &  22456  \\
    150   &  877    &  22065  \\
    200   &  847    &  21181  \\
    250   &  847    &  20389  \\
    300   &    &    \\
    350   &    &    \\
    400   &    &    \\
    450   &    &    \\
    500   &    &    \\
    550   &    &    \\
    \bottomrule
  \end{tblr}
\end{table}

Siehe \autoref{fig:plot} und \autoref{tab:tabelle}!


