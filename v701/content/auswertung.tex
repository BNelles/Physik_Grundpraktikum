\section{Auswertung}
\label{sec:Auswertung}

\subsection{Energieverlust und Reichweite von Alpha-Strahlung}
\begin{table}[H]
  \centering
  \caption{Messreihe bei einem Abstand von $\qty{6}{\centi\meter}$.}
  \label{tab:tabelle}
  \sisetup{table-format=1.1, per-mode=reciprocal}
  \begin{tblr}{
      colspec = {S S S},
      row{1} = {guard, mode=math},
    }
    \toprule
    p \mathbin{/} \unit{\milli\bar} & channel &  Pulse & \\
    \midrule
    0     &  871  &   19898 \\
    50    &  791  &   20435 \\
    100   &  808  &   17321 \\
    150   &  787  &   17543 \\
    200   &  760  &   17198 \\
    250   &  702  &   16791 \\
    300   &  718  &   15904 \\
    350   &  624  &   15343 \\
    400   &  732  &   1604  \\
    450   &  724  &   21    \\
    500   &  726  &   1     \\
    550   &       &   0     \\
    \bottomrule
  \end{tblr}
\end{table}

\begin{table}[H]
  \centering
  \caption{Messreihe bei einem Abstand von $\qty{5}{\centi\meter}$.}
  \label{tab:tabelle}
  \sisetup{table-format=1.1, per-mode=reciprocal}
  \begin{tblr}{
    colspec = {S S S},
    row{1} = {guard, mode=math},
    }
    \toprule
    p \mathbin{/} \unit{\milli\bar} & channel &  Pulse & \\
    \midrule
    0     &  1023   &  23373  \\
    50    &  919    &  22738  \\
    100   &  875    &  22456  \\
    150   &  877    &  22065  \\
    200   &  847    &  21181  \\
    250   &  847    &  20389  \\
    300   &    &    \\
    350   &    &    \\
    400   &    &    \\
    450   &    &    \\
    500   &    &    \\
    550   &    &    \\
    \bottomrule
  \end{tblr}
\end{table}



\subsection{Statistik des Zerfalls}

In der Tabelle \ref{tab:Statistik} ist die Anzahl der Alpha-Teilchen, welche innerhalb von 10 Sekunden auf den Detektor auftreffen, für 100 Durchführungen eingetragen. 



\begin{table}[H]
    \centering
    \caption{In dieser Tabelle sind die Messwerte für die Statistik des Zerfalls aufgeführt. $n$ ist hier die Nummer der Durchführung und $z$ die gemessene Zahl der Alphateilchen.} 
    \label{tab:Statistik}
    %\sisetup{table-format=1.1, per-mode=reciprocal}
    \begin{minipage}[t]{0.2\linewidth}
    \begin{tblr}[t]{
        colspec = {S[table-format=3.0] S[table-format=4.0]},
        row{1} = {guard, mode=math},
      }
      \toprule
      n  & z  \\
      \midrule
        1   &   1768  \\     
        2   &   1777  \\    
        3   &   1892  \\
        4   &   1601  \\
        5   &   1848  \\
        6   &   1721  \\
        7   &   1816  \\
        8   &   1922  \\
        9   &   1855  \\
       10   &   1762  \\
       11   &   1910  \\
       12   &   1845  \\
       13   &   1776  \\
       14   &   1928  \\
       15   &   1892  \\
       16   &   1850  \\
       17   &   1784  \\
       18   &   1874  \\
       19   &   1974  \\
       20   &   1827  \\
       21   &   1773  \\
       22   &   1883  \\
       23   &   1793  \\
       24   &   1711  \\
       25   &   1726  \\
      \bottomrule
    \end{tblr}
  \end{minipage}
  \hfill
  \begin{minipage}[t]{0.2\linewidth}
    \begin{tblr}[t]{
      colspec = {S[table-format=3.0] S[table-format=4.0]},
      row{1} = {guard, mode=math},
    }
    \toprule
    n  & z  \\
    \midrule
    26   &   1891  \\
    27   &   1797  \\
    28   &   1846  \\
    29   &   1817  \\
    30   &   1896  \\
    31   &   1909  \\
    32   &   1803  \\
    33   &   1940  \\
    34   &   1888  \\
    35   &   1895  \\
    36   &   1827  \\
    37   &   1927  \\
    38   &   1754  \\
    39   &   1794  \\
    40   &   1742  \\
    41   &   1759  \\
    42   &   1774  \\
    43   &   1824  \\
    44   &   1957  \\
    45   &   1833  \\
    46   &   1842  \\
    47   &   1903  \\ 
    48   &   1822  \\
    49   &   1842  \\
    50   &   1740  \\
    \bottomrule
  \end{tblr}
\end{minipage}
\hfill
\begin{minipage}[t]{0.2\linewidth}
  \begin{tblr}[t]{
    colspec = {S[table-format=3.0] S[table-format=4.0]},
    row{1} = {guard, mode=math},
  }
  \toprule
  n  & z  \\
  \midrule
  51   &   1873  \\
  52   &   1793  \\
  53   &   1938  \\
  54   &   1776  \\
  55   &   1819  \\
  56   &   1954  \\
  57   &   1926  \\
  58   &   1871  \\
  59   &   1766  \\
  60   &   1846  \\
  61   &   1773  \\
  62   &   1747  \\
  63   &   1753  \\
  64   &   1783  \\
  65   &   1923  \\
  66   &   1828  \\
  67   &   1836  \\
  68   &   1759  \\
  69   &   1772  \\
  70   &   1796  \\
  71   &   1799  \\
  72   &   1827  \\
  73   &   1799  \\
  74   &   1792  \\
  75   &   1784  \\
  \bottomrule
\end{tblr}
\end{minipage}
\hfill
\begin{minipage}[t]{0.2\linewidth}
  \begin{tblr}[t]{
    colspec = {S[table-format=3.0] S[table-format=4.0]},
    row{1} = {guard, mode=math},
  }
  \toprule
  n  & z  \\
  \midrule
  76   &   1822  \\
  77   &   1835  \\
  78   &   1790  \\
  79   &   1691  \\
  80   &   1903  \\
  81   &   1710  \\ 
  82   &   1881  \\
  83   &   1704  \\
  84   &   1894  \\
  85   &   1935  \\
  86   &   1956  \\
  87   &   1836  \\
  88   &   1914  \\
  89   &   1762  \\
  90   &   1868  \\ 
  91   &   1676  \\
  92   &   1772  \\
  93   &   1892  \\
  94   &   1876  \\
  95   &   1757  \\
  96   &   1880  \\
  97   &   1879  \\
  98   &   1870  \\
  99   &   1706  \\
 100   &   1836  \\
  \bottomrule
\end{tblr}
\end{minipage}

\end{table}


  

  Der Mittelwert der Teilchenzahl ist $\bar{z}=1827.08(72.22)$.
  Mit diesem Wert und seiner Standardabweichung kann nun eine Poisson-Verteilung und eine Gauß Verteilung simuliert werden.
  Diese sind in Abbildung \ref{fig:Statistik} aufgezeichnet.
  
  \begin{figure}[H]
    \includegraphics[width=\textwidth]{build/Statistik.pdf}
    \caption{Hier ist die Verteilung der Messwerte in blau, die Poisson-Verteilung in gelb und die Gauß-Verteilung in rot eingezeichnet.
    Dabei ist $z$ die Teilchenzahl und $N$ die Häufigkeit, mit der diese vorkommen.}
    \label{fig:Statistik}
  \end{figure}




