\section{Theorie}
\label{sec:Theorie}

Alphateilchen sind Heliumkerne, die in einem radioaktiven Zerfall entstehen. Die Teilchen sind verhältnismäßig
schwer und zweifach positiv geladen, wodurch sie stark mit ihrer Umgebung wechselwirken. Wenn sie in einen
Halbleiter eintreten, erzeugen sich in dem neutralen Bereich des Halbleiters Elektronenlöcherpaare. Wenn auf dem Halbleiter eine Spannung, die gegen die Betribsrichtung gerichtet ist anliegt
entsteht dadurch ein Strompuls. Durch die vielen Interaktionen mit umliegender Materie haben Alpha-Teilchen in der Regel
nur eine geringe Reichweite $R$, die sie nach
\begin{equation}
    R=\int_0^{E_\alpha}(\frac{\symup{d}E_\alpha}{\symup{d}x})^{-1}\symup{d}E_\alpha  
    \label{eqn:Weglänge}
\end{equation}
\noindent zurücklegen können. Dabei wird der Energieverlust $\frac{\symup{d}E_\alpha}{\symup{d}x}$ eines Alphateilches
über die mittlere Weglänge durch
\begin{equation}
    \frac{\symup{d}E_\alpha}{\symup{d}x}=\frac{z^2 e^4}{4\pi \epsilon_0 m_e}\frac{nZ}{v^2}ln(\frac{2m_ev^2}{E_I})
    \label{eqn:bethe}
\end{equation}
\noindent beschrieben.
In Abhängigkeit des Drucks wird dann eine effektive Weglänge $x$ mit der Proportionanlität
\begin{equation}
    x=x_0\frac{p}{p_0}
    \label{eqn:druck}
\end{equation}
eingeführt, die den Abstand zwischen der Quelle und dem Detektor nimmt und mit dem verhältnis des Drucks in der Kammer
mit dem Atmosphärendruck in Beziehung setzt.
\cite{sample}
