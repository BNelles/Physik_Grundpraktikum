\section{Theorie}
\label{sec:Theorie}

Alphateilchen sind Heliumkerne, die in einem radioaktiven Zerfall entstehen. Die Teilchen sind verhältnismäßig
schwer und zweifach positiv geladen, wodurch sie stark mit ihrer Umgebung wechselwirken. Wenn sie in einen
Halbleiter eintreten, erzeugen sich in dem neutralen Bereich des Halbleiters Elektronenlöcherpaare, da der Heliumkern
zwei Elektronen noch aufnehmen kann. Wenn auf dem Halbleiter eine Spannung, die gegen die Betribsrichtung gerichtet ist anliegt
entsteht dadurch ein Strompuls. Durch die vielen Interaktionen mit umliegender Materie haben Alpha-Teilchen in der Regel
nur eine geringe mittlere Weglänge $R$, die sie nach
\begin{equation}
    R=\int_0^{E_\alpha}(\frac{\symup{d}E_\alpha}{\symup{d}x})^(-1)\symup{d}E_\alpha  
    \label{eqn:Weglänge}
\end{equation}
\noindent zurücklegen können. Haben die Alphateilchen eine Energie, die groß genug ist, dann wird $\frac{\symup{d}E_\alpha}{\symup{d}x}$
durch
\begin{equation}
    \frac{\symup{d}E_\alpha}{\symup{d}x}=\frac{z^2 e^4}{4\pi \epsilon_0 m_e}\frac{nZ}{v^2}ln(\frac{2m_ev^2}{E_I})
    \label{eqn:bethe}
\end{equation}
\noindent beschrieben.
In Abhängigkeit des Drucks wird dann eine freie Weglänge $x$ mit der proportionanlität
\begin{equation}
    x=x_0\frac{p}{p_0}
    \label{eqn:druck}
\end{equation}
modifiziert.
\cite{sample}
