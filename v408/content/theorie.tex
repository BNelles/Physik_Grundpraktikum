\section{Theorie}
\label{sec:Theorie}

    \noindent Bei Linsen wird generell zwischen Sammellinsen und Zerstreuungslinsen unterschieden. Sammel und Streuungslinsen unterscheiden sich dadrin, dass die Brennweite $f$ bei Sammellinsen positiv ist und bei Zerstreuungslinsen negativ.
    Zusätzlich werden auch zwei Hauptebenen definiert, an denen die Brechung mathematisch modelliert wird. Im Fall einer dünnen Linse fallen die beiden Hauptebenen auf die Mittelebene zusammen. Der Abbildungsmaßstab $V$, der das Verhältnis der Bildgröße $B_g$
    zu der Gegenstandsgröße $G_g$ wiedergibt, wird in diesem Fall auch zu
    \begin{equation}
        V=B_g/G_g=b/g \text{,}
        \label{eqn:abbildung}
    \end{equation}
    \noindent wobei $b$ der Abstand zwischen dem Bild und der Mittelebene ist und $g$ der Abstand vom Gegenstand zur Mittelebene.
    Des weiteren gilt bei solchen Linsen für die Brennweite die Beziehung
    \begin{equation}
        1/f=1/b+1/g \text{,}
        \label{eqn:brenn1}
    \end{equation}
    \noindent wenn das Bild in der Bildebene scharf erscheint.


\cite{sample}
