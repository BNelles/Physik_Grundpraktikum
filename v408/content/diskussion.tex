\section{Diskussion}
\label{sec:Diskussion}
\subsection{Brennweitenbestimmung durch Messung von Gegenstands- und Bildweite}
Mithilfe der Abweichung die in \ref{tab:brenn} berechnet wurden ist erkennbar, dass diese mit steigender Gegenstandsweite auch größer wird.

\noindent Bei der Bestimmung der Brennweite ist das Ergebnis $f \bar=\qty{10.178(0.335)}{\centi\meter}$. 
Die Abweichung dieses Ergebnisses von Theoriewert beträgt mit 
\begin{equation}
    A=\frac{E-T}{T} \cdot \qty{100}{\percent}
    \label{eqn:Abweichung}
\end{equation}
$A_f=\qty{1.8(3.3)}{\percent}$.
Mögliche Fehlerquellen sind hier Messfehler bei der Bestimmung der Gegenstands- und Bildweite, sowie Abweichungen bei der tatsächlichen Brennweite der Linse.

\noindent Die Graphisch bestimmten werte sind $f_{g_1}=x=\qty{9(1)}{\centi\meter}$ und $f_{g_2}=y=\qty{11(1)}{\centi\meter}$ befindet.
Da diese Beiden Werte nich identisch sind, liegt ein Fehler vor.
Für die einzelnen Werte ergeben sich mit Formel \ref{eqn:Abweichung} die Abweichungen $A_{fg1}=\qty{-10(10)}{\percent}$ und $A_{fg2}=\qty{10(10)}{\percent}$
Dieser lässt sich in den zuvor genannten Fehlerquellen begründen, sowie dem Ablesefehler, der beim Ablesen von der Abbildung noch dazu kommt.

\subsection{Brennweitenbestimmung mit der Methode von Bessel}
Der durch diese Methode bestimmte Wert für die Brennweite ist $f_{B}=\qty{9.80(0.06)}{\centi\meter}$.
Bei diesem beträgt die Abweichung zum Theoriewert $A_{f_B}=\qty{2.0(0.6)}{\percent}$.
Die gemessenen Brennweiten mit Farbfiltern sind $f_b=\qty{9.75(0.06)}{\centi\meter}$ und $f_r=\qty{9.82(0.08)}{\centi\meter}$.
Dazu sind $A_{f_Bb}=\qty{-2.5(0.6)}{\percent}$ und $A_{f_Br}=\qty{-1.8(0.8)}{\percent}$ die entsprechenden Abweichungen zum Theoriewert.
Es ist erkennbar, dass Die Brennweite mit blauem Filter niedriger ist, als die ohne Filter.
Zudem ist die Brennweite mit rotem Filter höher, als bei der Messung ohne Filter.
Dieses Ergebnis wird auch so durch chromatische Abberration vorausgesagt.
Jedoch ist durch die Fehler, die sich in derseben Größenordnung wie die Abweichung zu dem Filterlosen Licht befinden, fraglich, inwiefern das Ergebnis wirklich die chromatische Abberration belegt.


\subsection{Brennweitenbestimmung mit der Methode von Abbe}
Die Brennweite, die sich durch die Methode von Abbe für das Linsensystem ergibt, ist $f_A=\qty{20.5(1.0)}{\centi\meter}$.
Dies ist insofern zu erwarten, dass die Brennweite des Linsensystems größer ist, als die der einzelnen Sammellinse mit $f=\qty{10}{\centi\meter}$,
da das Licht nun gestreut wird, bevor es die Sammellinse erreicht und sich dadurch der Brennpunkt nach hinten verschiebt.
Berechnet wurde ein Wert von $f_{A_t}=\qty{22.22}{\centi\meter}$ und die mit Formel \ref{eqn:Abweichung} ausgerechnete Abweichung beträgt $A_{f_A}=\qty{-8(5)}{\percent}$.
Dies könnte, neben möglichen Fehlern während der Messung, auch durch die Näherungen bei der theoretischen Berechnung kommen.
In dieser wird nämlich sphärische Abberation nicht berücksichtigt.
