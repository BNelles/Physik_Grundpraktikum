

\section{Theorie}
\label{sec:Theorie}
Beim Franck-Hertz Versuch werden die aus dem Glühdraht austretenden Elektronen durch die Spannungsdifferenz zwischen Glühdraht und Beschleunigungselektrode beschleunigt, bis sie die nötige kinetische Energie haben,
um ein Quecksilber Atom anzuregen.
Diese Elektronen können dabei elastisch oder inelastisch stoßen.
Bei einem elastischen Stoß ändert sich die Energie der gestoßenen Elektronen nicht, sondern nur die Richtung dieser.
Die Energie, die ein Elektron bei einem inelastischen Stoß abgibt, beträgt
\begin{equation}
    \delta E=\frac{1}{2} m_0 \cdot v_{vor}^2-\frac{1}{2} m_0 \cdot v_{nach}^2 \. .
    \label{eqn:Energie}
\end{equation}
Dabei wird die Gegenfeldmethode benutzt, um die Energie der schnellsten Elektronen zu bestimmen.
Es wird zwischen Beschleunigerelektrode und Auffängerelektrode eine Gegenspannung $U_A$ angelegt.
Nach der Beschleunigung haben die Elektronen die Energie
\begin{equation}
    \frac{1}{2} m_0 \cdot v_{vor}^2=e_0 U_B
    \label{eqn:Beschleunigung}
\end{equation}
und durch das Gegenfeld werden nur die Elektron mit
\begin{equation}
    \frac{1}{2} m_0 \cdot v_{z}^2 \>= e_0 U_A
    \label{frac:Gegenspannung}
\end{equation}
gemessen.

\cite{V604}
