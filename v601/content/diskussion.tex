\section{Diskussion}
\label{sec:Diskussion}

Der Verlauf der Franck-Hertz Kurve lässt sich in Abbildung \ref{fig:2} erkennen.
Sie unterscheidet sich von der theoretischen Franck-Hertz Kurve in Abbildung \ref{fig:Theorie}.
Die Unterschiede sind die, die in der Theorie durch die Fermi-Dirac Verteilung der Elektronen, den Dampfdruck und das Kontaktpotential erwartet werden.
Die Wellenlänge des Photons, welches von dem angeregten Atom emmitiert wird, beträgt $\lambda_e=\qty{250(12)}{\nano\meter}$.
Der tatsächliche Übergang besitzt eine Wellenlänge von $\lambda_t=\qty{253}{\nano\meter}$ \cite{HG}.
Damit ergitb sich mit der Formel 
\begin{equation}
    A_{et}=\Biggl|\frac{w_e-w_t}{w_t}\Biggr|
    \label{eqn:Abweichung}
\end{equation}
eine Abweichung von $A_{et}=\qty{1(5)}{\percent}$.
Mögliche Fehlerquellen bei der genauen Bestimmung sind die Energieverteilung der Elektronen und die Abhängigkeit des Stoßzeitpunktes vom Dampfdruck.
Für das Kontaktpotential wurde ein Wert von $K=\qty{1.2(0.09)}{\electronvolt}$ ermittelt.
Da dieses jedoch nur für die Verschiebung des Graphen sorgt und die Abstände der Form der Peaks nicht ändert, ist es nicht für die Abweichung veranwortlich.
