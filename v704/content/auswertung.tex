\section{Auswertung}
\label{sec:Auswertung}

\subsection{Der Gammastrahler}

\begin{table}[H]
  \centering
  \caption{Die gemessenen Zählraten $N$ nach der Absorption durch Eisen der Dicke $d$ nach der Zeit $t$.}
  \label{tab:tabelle}
  \sisetup{table-format=1.1, per-mode=reciprocal}
  \begin{tblr}{
      colspec = {S[table-format=2.1] S S[table-format=4.0]},
      row{1} = {guard, mode=math},
    }
    \toprule
    d \text{/}\unit{\milli\meter} & t \text{/} \unit{\second} & N \\
    \midrule
    0.0   &     60  &    7246\\
    0.5   &     60  &    6624\\
    1.0   &     60  &    6745\\
    1.5   &     60  &    6926\\
    2.0   &     60  &    6833\\
    2.5   &     60  &    6602\\
    3.0   &     60  &    6084\\
    3.5   &     60  &    5772\\
    4.0   &     60  &    5656\\
    4.5   &     60  &    5873\\
    5.0   &     60  &    6130\\
    5.5   &     60  &    5790\\
    6.0   &     60  &    6081\\
    10    &     60  &    4365\\
    20    &     60  &    2816\\
    \bottomrule
  \end{tblr}
\end{table}

\begin{figure}[H]
  \includegraphics{build/eisen.pdf}
  \caption{Der Logarithmus der Zählrate in $\unit{\per\second}$ gegen die Dicke des Eisens.}
  \label{fig:eisen}
\end{figure}

Aus der Regression von \ref{fig:eisen} lässt sich der Absorptionskoeffizient von Eisen bestimmen als $\mu=\qty{47(2.8)}{\per\meter}$. 
Der theoretische Wert für den Absorptionskoeffizienten ist hier $\mu_t=\qty{34.1}{\per\meter}$.


\begin{table}[H]
  \centering
  \caption{Die gemessenen Zählraten $N$ nach der Absorption durch Aluminium der Dicke $d$ nach der Zeit $t$.}
  \label{tab:tabelle}
  \sisetup{table-format=1.1, per-mode=reciprocal}
  \begin{tblr}{
      colspec = {S[table-format=2.1] S S[table-format=4.0]},
      row{1} = {guard, mode=math},
    }
    \toprule
    d \text{/}\unit{\milli\meter} & t \text{/} \unit{\second} & N \\
    \midrule
    2   &     60  &    6490\\
    4   &     60  &    5753\\
    6   &     60  &    5397\\
    8   &     60  &    5160\\
    10  &     60  &    4816\\
    12  &     60  &    4165\\
    14  &     60  &    4006\\
    16  &     60  &    3592\\
    18  &     60  &    3315\\
    20  &     60  &    2982\\
    \bottomrule
  \end{tblr}
\end{table}

\begin{figure}[H]
  \includegraphics{build/aluminium.pdf}
  \caption{Der Logarithmus der Zählrate in $\unit{\per\second}$ gegen die Dicke des Aluminiums.}
  \label{fig:alu}
\end{figure}

Aus der Regression von \ref{fig:alu} lässt sich der Absorptionskoeffizient von Eisen bestimmen als $\mu=\qty{42.4(1.4)}{\per\meter}$.
Der theoretische Wert beträgt hier $\mu_t=\qty{12.1}{\per\meter}$.

\subsection{Der Betastrahler}

\begin{table}[H]
  \centering
  \caption{Die gemessenen Zählraten $N$ nach der Absorption durch die Absorberplatten der Dicke $d$ nach der Messzeit $t$.}
  \label{tab:tabelle}
  \sisetup{table-format=1.1, per-mode=reciprocal}
  \begin{tblr}{
      colspec = {S[table-format=2.1] S S[table-format=4.0]},
      row{1} = {guard, mode=math},
    }
    \toprule
    d \text{/}\unit{\milli\meter} & t \text{/} \unit{\second} & N \\
    \midrule
    0    &   60   &   32130\\
    100  &   60   &   2130\\
    125  &   100  &   969\\
    153  &   120  &   1134\\
    160  &   140  &   773\\
    200  &   160  &   376\\
    253  &   180  &   171\\
    302  &   200  &   172\\
    338  &   250  &   188\\
    400  &   250  &   154\\
    444  &   250  &   163\\
    482  &   300  &   166\\
    \bottomrule
  \end{tblr}
\end{table}

\begin{figure}[H]
  \includegraphics{build/beta.pdf}
  \caption{Der Logarithmus der Zählrate in $\unit{\per\second}$ gegen die Dicke der Absorberplatten.}
  \label{fig:beta}
\end{figure}

Aus der Regression von \ref{fig:beta} lässt sich der Absorptionskoeffizient der Absorberplatten bestimmen als $\mu=\qty{2.57(0.14)e4}{\per\meter}$.
Aus dem Schnittpunkt der linearen Regression ergibt sich für $R_{max}=\qty{0.4(0.8)}{\milli\meter}$, wodurch $E_{max}=\qty{0.018(0.017)}{\mega\electronvolt}$
ist.