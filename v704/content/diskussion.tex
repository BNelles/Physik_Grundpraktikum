\section{Diskussion}
\label{sec:Diskussion}
\subsection{Gamma-Srahlung}
Die Absorbtionskurve für Gamma-Strahlung ist in \ref{fig:eisen} für Eisen und in \ref{fig:alu} für Aluminium gegeben.
Mithilfe dieser werden experimentelle Werte für die Absorbtionskoeffizienten bestimmt.
Für Eisen ist dieser Wert $\mu_{E_e}=\qty{47(2.8)}{\per\meter}$ und für Aluminium $\mu_{A_e}=\qty{42.4(1.4)}{\per\meter}$.
Die theoretisch bestimmten Werte sind $\mu_{E_t}=\qty{34.1}{\per\meter}$ und $\mu_{A_t}=\qty{12.1}{\per\meter}$.
Mithilfe von
\begin{equation}
    a=\biggl|\frac{W_e-W_t}{W_t}\biggr| \cdot 100\unit{\percent}
    \label{eqn:Abweichung}
\end{equation}
wird dann die Abweichung bestimmt.
Für Eisen ist diese $a_E=\qty{38(8)}{\percent}$ und für Aluminium $a_A=\qty{250(12)}{\percent}$. 
Mögliche Fehlerquellen sind zum einen Fehler bei der Aufnahme der Werte, da die Platten aufgrund von Problemen mit der 
Apparatur nicht vernünftig aufgestellt werden konnten, sondern an zwei Metallbalken angelehnt werden mussten.
Dadurch standen diese nicht immer gerade, wordurch die tatsächliche hier relevante Dicke des Materials von der der
 Platten abweichen könnte.
Zudem war die Halterung des Gamma-Strahlers nicht komplett fixiert, was auch zu Abweichungen führen kann.

\subsection{Beta-Strahlung}
Der Graph in \ref{fig:beta} stellt die Absorbtionskurve von Beta-Strahlung dar.
Hier wird für den Absorbtionskoeffizienten $\mu=\qty{2.57(0.14)e4}{\per\meter}$ bestimmt.
Die errechnete maximale Energie beträgt $E_{max}=\qty{0.018(0.017)}{\mega\electronvolt}$.
Hier sind mögliche Fehler die geringe Anzahl an Messwerten, die bei der Regression bei Beta-Strahlung nach dem Knick,
also der Bereich, in dem nur noch Untergrund gemessen wurde.