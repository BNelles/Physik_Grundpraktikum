\section{Durchführung}
\label{sec:Durchführung}
\subsection{Aufbau}
Auf einem Luftkissen, in einem Helmholtz-Spulenpaars, wurde eine Billiardkugel plaziert.
In dieser befindet sich ein Permanentmagnet.
Das Luftkissen sorgt dafür, dass sich die Kugel hier annähernd reibungsfrei drehen kann.
An dem Aufbau befindet sich außerdem ein, auf die Kugel gerichtetes, Stroboskop mit einer Frequenz von $\qty{5}{\hertz}$.
An der Kugel befindet sich ein kleiner Stab, auf dem ein weißer Punkt aufgemalt ist.


\subsection{Bestimmung des magnetischen Momentes über Präzession}
Hier wird die Kugel in Rotation versetzt.
Danach wird diese um einen kleinen Winkel ausgelenkt.
Mithilfe des Stroboskopes soll dann die Drehungsfrequenz der Kugel an die des Stroboskopes angepasst werden.
Dabei wird, nach dem Andrehen der Kugel, durch das Aufleuchten des weißen Punktes ermittelt, ob die Kugel sich schnell genug dreht.
Wenn der Punkt sich augenscheinlich mit der Drehrichtung der Kugel dreht, ist die Drehungsfrequenz größer als die des Stroboskopes.
Dann wird gewartet, bis der Punkt so erscheint, als würde er stehenbleiben, da dann die richtige Frequenz erreicht wurde.
Sofort wird das äußere Magnetfeld eingeschaltet und mithilfe einer Stoppuhr, die Umlaufdauer gemessen.
Die Messung wird jeweils drei mal für fünf verschiede Magnetfeldstärken durchgeführt. 

\subsection{Bestimmung des magnetischen Momentes über Gravitation}
In den Stab der Kugel wird eine Aluminiumstange gesteckt, an der sich ein verschiebbares Gewicht befindet.
Der Abstand des Schwerpunktes des Gewichts vom Permanentmagneten zu dem Mittelpunkt der Kugel ist $r$.
Hier werden 9 verschiedene Werte von $r$ eingestellt. 
Daraufhin wird für jedes $r$ die Magnetfeldstärke so angepasst, dass das durch das Magnetfeld erzeugte Drehmoment das der Gravitation genau kompensiert.
Dabei wird die Stromstärke des Stroms $I_H$, der durch das Helmholtz-Spulenpaar fließt, gemessen.

\subsection{Bestimmung des magnetischen Momentes über die Schwingungsdauer}
Bei der dritten Art der Bestimmung des magnetischen Momentes wird die Aluminiumstange wieder an der Kugel befestigt, aber diesmal ohne Gewicht.
Die Kugel wird nun innerhalb des Helmholtz-Spulenpaars, bei eingeschaltem Magnetfeld, um einen kleinen Winkel ausgelenkt.
Dadurch kann die daurauf Folgende Schwingung annähernd als harmonischer Oszillator beschrieben werden.
FÜr 9 verschiedene Stromstärken $I_H$ wird dann die Dauer von 10 Schwingungsperioden bestimmt.