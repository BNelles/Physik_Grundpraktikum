\section{Diskussion}
Bei der Bestimmung des Dipolmomentes des Permanentmagneten durch Präzession ergibt sich $|\vec{m_P}|=\qty{0.373(0.018)}{\ampere\meter\squared}$.
Für die Bestimmung mit Gravitation wird ein Wert von $|\vec{m_G}|=\qty{1.07(0.03)}{\ampere\meter\squared}$ berechnet.
Aus der Methode mit Schwingung erfolgt $|\vec{m_S}|=\qty{19.7(1.6)}{\ampere\meter\squared}$.

Zwischen den einzelnen Werten wird die Abweichung mit 
\begin{equation}
    |A_{1,2}|=\frac{|\vec{m_1}|-|\vec{m_2}|}{|\vec{m_2}|}
\end{equation}
berechnet.

Dadurch ergibt sich für die Abweichung von $|\vec{m_G}|$ zu $|\vec{m_P}|$ $A_{G,P}=\qty{187(0.16)}{\percent}$ und bei der von $|\vec{m_S}|$ zu $|\vec{m_P}|$ $A_{S,P}=\qty{5200(5)}{\percent}$.
Die Abweichung von Schwingungs- zur Gravitationsmethode $A_{S,G}=\qty{1740(1.6)}{\percent}$.

Hier wird deutlich, dass die Abweichung von Gravitation und Präzession die geringste ist.
Die Abweichung die hierbei vorliegt ist dennoch groß, was zum Teil damit zu tun haben kann, dass für die Messung über die 
Präzession ein anderer Aufbau verwendet wurde.

Das ist auch ein Grund für die Abweichung $A_{S,P}$.
Jedoch ist diese deutlich größer, da auch die Abweichung zwischen Schwingung und Gravitation zur Bestimmung deutlich größer als $A_{G,P}$ ist.

Dies kann damit begründet werden, dass der Aufbau, mit dem das Dipolmoment über die Gravitation und 
die Schwingungsdauer bestimmt wurde, Probleme mit dem Luftkissen hatte.
Dies war für die Bestimmung mit Gravitation nicht besonders relevant, da hier keine Zeit gemessen wurde und kurze Aussetzer des Luftkissens nicht dafür gesorgt haben, dass Werte für die Stromstärke abgelesen werden können.
Bei der Bestimmung durch die Schwingungsdauer sollte dies aber relevanter sein.
Beispielsweise unterbrach die Schwingung der Kugel aber manchmal für 
einen kurzen Moment, bevor diese die Bewegung fortsetzte.
Dass die Methode der Bestimmung durch Schwingung im Vergleich zu den anderen beiden noch ungenauer ist, ist auch nach dem Graphen in Abbildung \ref{fig:schwing} zu erwarten, da die Form der Werte nicht linear ist, wie hier angenommen.
\label{sec:Diskussion}
