\section{Auswertung}
\label{sec:Auswertung}

\subsection{Wheatstonsche Brückenschaltung}
Für $R_2$ wurde ein Widerstand mit $R_2=\qty{332(9,96)}{\ohm}$ verwendet.
Der Fehler der Bauteile wird im Folgenden immer als $\pm\qty{3}{\percent}$ angenommen.
Der zu bestimmende Widerstand war Wert 13.
Durch das varieren der Widerstände $R_3$ und $R_4$, bis die Brückenspannung den Wert 0 annimmmt, wurden für diese bei der ersten Messung die Werte
$R_3=\qty{490}{\ohm}$ und $R_4=\qty{510}{\ohm}$ bestimmt, mit dem Verhältnis $\frac{R_3}{R_4}=0,961\pm0,005$.
Für den Fehler des Verhältnisses wurde $\pm\qty{0,5}{\percent}$ angenommen.
Mit Gleichung \ref{eqn:widerstand} ergibt sich $R_{13}=\qty{318,98}{\ohm}$.
Der Fehler dieser Größe wird mit Formel \ref{eqn:fehlerfortpflanzung} berechnet.
Damit ergibt sich hier $\increment R_{13}=\sqrt{ \Bigl(\frac{R_3}{R_4}\Bigr)^2 (\increment R_2)^2+(R_2)^2 \Bigl(\increment \frac{R_3}{R_4}\Bigr)^2}$.
Demnach gilt $R_{13}=\qty{318,98(9,71)}{\ohm}$.
Dies wurde für einen Weiteren unbekannten Widerstand, Wert 11, durchgeführt.
Hier ergibt sich $R_3=\qty{596}{\ohm}$, $R_4r=\qty{404}{\ohm}$ und $\frac{R_3}{R_4}=1,475\pm0,007$.
Daraus folgt mit Formel \ref{eqn:widerstand} und \ref{eqn:fehlerfortpflanzung} $R_{11}=\qty{490(15)}{\ohm}$.

\subsection{Kapazitive Brückenschaltung}
Für $R_2$ wird hier der selbe Wert wie bei der Wheatstonschen Brücke genutzt und die Kapizität des Kondensators $C_2$ ist $C_2=\qty{750(22,5)}{\nano\farad}$.
Die unbekannte Kapazität war hier Wert 8.
Durch das varieren von $R_3$ und $R_4$ bis ein Minumum gefunden wurde, wurden die Werte $R_3=\qty{675}{\ohm}$ und $R_4=\qty{325}{\ohm}$ aufgenommen.
Das Verhältnis der Werte beträgt $\frac{R_3}{R_4}=2,08\pm0,01$. 
Mit Gleichung \ref{eqn:widerstand} und \ref{eqn:fehlerfortpflanzung} ergibt sich $R_8=\qty{690(21)}{\ohm}$.
Die Kapazität $C_8$ kann mit Gleichung \ref{eqn:kapazität} und der Fehler mit Gleichung \ref{eqn:fehlerfortpflanzung} berechnet werden.
Der Fehler ist also $\increment C_{8}=\sqrt{ \Bigl(\frac{R_4}{R_3}\Bigr)^2 (\increment C_2)^2+(-C_2 \frac{R_4^2}{R_3^2})^2 \Bigl(\increment \frac{R_3}{R_4}\Bigr)^2}$.
Damit gilt $C_8=\qty{361(11)}{\farad}$.\\

Dies wird analog mit Wert 9 durchgeführt.
Damit ist $R_3=\qty{595}{\ohm}$ und $R_4=\qty{405}{\ohm}$.
Einsetzen in Formel \ref{eqn:widerstand} und \ref{eqn:fehlerfortpflanzung} ergibt $R_9=\qty{488(15)}{\ohm}$.
Die Kapazität hat nach Einsetzen in Formel \ref{eqn:kapazität} und \ref{eqn:fehlerfortpflanzung} den Wert $C_9=\qty{511(7)}{\ohm}$.




\begin{table} 
  \centering
  \caption{Eine Beispieltabelle mit Messdaten.}
  \label{tab:tabelle}
  \sisetup{table-format=1.1, per-mode=reciprocal}
  \begin{tblr}{
      colspec = {S[table-format=3.0] S[table-format=2.1] S},
      row{1} = {guard, mode=math},
      vline{4} = {2}{-}{text=\clap{$\pm$}},
    }
    \toprule
    U \mathbin{/} \unit{\volt} & I \mathbin{/} \unit{\micro\ampere} & \SetCell[c=2]{c} N \mathbin{/} \unit{\per\second} & \\
    \midrule
    360 & 0.1 & 98.3 & 0.9 \\
    400 & 0.2 & 99.8 & 1.0 \\
    420 & 0.2 & 99.1 & 0.9 \\
    \bottomrule
  \end{tblr}
\end{table}

%Siehe \autoref{fig:plot} und \autoref{tab:tabelle}!
