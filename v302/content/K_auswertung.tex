\subsection{Klirrfaktor}

\begin{table}
    \centering
    \caption{Brückenspannung in Abhängigkeit von der Frequenz}
    \label{frequenzen}
    \begin{tblr}{
        colspec={S S},
        row[1]={guard, mode=math}
    }
    \toprule
    f[Hz] & U_\text{Br}[mV] \\
    \midrule
    50    &  230    \\
    100   &  100    \\  
    150   &  15     \\
    200   &  50     \\
    250   &  100    \\   
    300   &  130    \\
    350   &  160    \\
    400   &  200    \\
    450   &  200    \\
    500   &  220    \\
    1000  &  280    \\
    1500  &  300    \\
    2000  &  300    \\
    2500  &  300    \\
    3000  &  310    \\
    3500  &  310    \\
    4000  &  310    \\
    4500  &  310    \\
    5000  &  310    \\
    \bottomrule
    \end{tblr}
\end{table}

Für die Eingangsspannung gilt hier $U_s=\qty{1}{V}$, die Kapazität der Kondensatoren beträgt jeweils $\qty{994}{\nano\farad}
\text{und} \qty{992}{\nano\farad}$ und wird hier auf $\qty{993}{\nano\farad}$ gemittelt, da eigentlich gelten müsse, dass 
diese gleich sind, dies aber nicht realisierbar war. Desweiteren betrtägt $R'=\qty{332}{\ohm} \text{und} R=\qty{1}{\kilo\ohm}$.
Die Bezeichnungen dieser Größen finden sich in \ref{fig:Wien} wieder.\\
Zur bestimmung des Klirrfaktors wird hier nur die zweite Oberwelle betrachtet. In Gleichung \ref{} 

Die Frequenz, für die die Brückenspannung minimiert wird, $f_0$ ist hier bei $\qty{150}{Hz}$ erreicht.


\begin{figure}
    \centering
    \includegraphics{../frequenz.pdf}
    \caption{Brückenspannung}
    \label{fig:frequenzverhältnis}
\end{figure}

