\section{Theorie}
\label{sec:Theorie}
    \subsection{allgemeine Formeln}
    Das ohmsche Gesetz beschreibt den Zusammenhang zwichen der Spannung an einem Leiter mit der Stromstärke.
    \begin{equation}
        U=R \cdot I
        \label{eqn:ohm}
    \end{equation}\\
    Die kirchhoffschen Gesetze beschreiben das Verhalten von Strom in einem geschlossenen Stromkreis.
    Die Knotenregel besagt, dass alle einem Knoten hinzugefügten Ladungen gleich der abgegebenen Ladungen sein müssen.
    \begin{equation}
        \sum_{i=1}^{\symup{n}}{I_i}
        \label{eqn:knoten}
    \end{equation}\\
    Die Maschenregel besagt, dass die Summe aller einzelnen Spannungen in einer Masche gleich 0 ist. 
    Das liegt daran, dass die zugeführte und abgegebene elekztrische Arbeit gleich groß sein muss.
    \begin{equation}
        \sum_{i=1}^{\symup{n}}{U_{0,i}} - \sum_{j=1}^{\symup{m}}{U_{ab,j}} =0
        \label{eqn:masche}
    \end{equation}\\

    \subsection{Wheatstonsche Brückenschaltung}
    Die Wheatstonsche Brücke ist eine Schaltung, die genutzt wird, um ohmsche Widerstände zu messen.
    Dabei ist eine Spannung an zwei parallel geschalteten Paaren aus Widerständen angeschlossen, wobei nach dem ersten Widerstand jeweils ein Kabel an einem Oszilloskop angeschlossen ist.
    %\ref{Bild}
    Drei der Widerstände sind bekannt, der andere ist zu bestimmen.
    Dies erfolgt mit der Formel
    \begin{equation}
        R_x=R_2 \frac{R_3}{R_4}
        \label{eqn:widerstand}
    \end{equation}^

    \subsection{Kapazitätsmessbrücke}
    Hier werden, im Gegensatz zu der Wheatstonsche Brückenschaltung, vor die beiden Widerstände auf der linken Seite jeweils ein Kondensator in Reihe geschaltet.
    Dabei hat der nach der Verbindung zum Oszilloskop eine bekannte Kapazität, wohingegen der der bei dem unbekannten Widerstand installierte eine zu bestimmende Kapazität besitzt.
    Den unbekannten Widerstand berechnet man mit Gleichung \ref{eqn:widerstand}.
    Die unbekannte Kapazität mit 
    \begin{equation}
        C_x=C_2 \frac{R_4}{R_3}
        \label{eqn:kapazität}
    \end{equation}

    \subsection{Induktivitätsmessbrücke}
    Der Aufbau der Induktivitätsmessbrücke ist analog zu dem der Kapazitätsmessbrücke, außer dass die Kondensatoren durch Spulen ausgetauscht werden.
    Damit gilt für die Berechnung des ohmschen Widerstandes wieder Gleichung \ref{eqn:widerstand} und die Induktivität der Spule, die bestimmt werden soll, wird mit 
    \begin{equation}
        L_x=L_2 \frac{R_4}{R_3}
        \label{eqn:Induktivität}
    \end{equation}

    \subsection{Maxwell-Brücke}
    Die Maxwell-Brücke wird ebenfalls zur Bestimmung der Induktivität einer Spule benutzt.
    Sie besitzt aber keine Spule $L_2$ in Reihe zu $R_2$, sondern stattdessen einen Kondensator, der parallel zu $R_4$ geschaltet ist.
    $R_4$ ist nun ein verstellbarer Widerstand. Das ist damit zu begründen, dass ein Kondensator mit geringen Verlusten einfacher zu realisieren ist, als eine Spule mit wenig Verlusten.
    Die Berechnung von $L_x$ erfolgt nun über
    \begin{equation}
        L_x=R_2 R_3 C_4
    \end{equation}
%\cite{sample}

