\section{Theorie}
\label{sec:Theorie}
    Das ohmsche Gesetz beschreibt den Zusammenhang zwichen der Spannung an einem Leiter mit der Stromstärke.
    \begin{equation}
        U=R \cdot I
        \label{eqn:ohm}
    \end{equation}\\
    Die kirchhoffschen Gesetze beschreiben das Verhalten von Strom in einem geschlossenen Stromkreis.
    Die Knotenregel besagt, dass alle einem Knoten hinzugefügten Ladungen gleich der abgegebenen Ladungen sein müssen.
    \begin{equation}
        \sum_{i=1}^{\symup{n}}{I_i}
        \label{eqn:knoten}
    \end{equation}
    Die Maschenregel besagt, dass die Summe aller einzelnen Spannungen in einer Masche gleich 0 ist. 
    Das liegt daran, dass die zugeführte und abgegebene elekztrische Arbeit gleich groß sein muss.
    \begin{equation}
        \sum_{i=1}^{\symup{n}}{U_{0,i}} - \sum_{j=1}^{\symup{m}}{U_{ab,j}} =0
    \end{equation}
\cite{sample}
