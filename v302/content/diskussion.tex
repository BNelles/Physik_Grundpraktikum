\section{Diskussion}
    \subsection{Wheatstonesche Brückenschaltung}
    Für den Widerstand, der mit der Wheatstoneschen Brücke bestimmt wurde, ergibt sich ein Wert von $R_{13}=\qty{318,98(9,71)}{\ohm}$.
    Der tatsächliche Wert von Wert 13 wird hier als $R_{13}'$ definiert und beträgt $R_{13}'=\qty{319}{\ohm}$.
    Für den Quotienten von bestimmten Wert und Realwert gilt $\frac{R_{13}}{R_{13}'}=0,9999373\pm0,0304389$.

    Für Wert 11 ergibt sich mit $R_{11}=\qty{490(15)}{\ohm}$ und $R_{11}'=\qty{489,9}{\ohm}$ $\frac{R_{11}}{R_11'}=1,002\pm0,031$.

    \subsection{Kapazitätsmessbrücke}
    Der, mit der Kapazitätsmessbrücke bestimmte Widerstand, hat den Wert $R_8=\qty{690(21)}{\ohm}$.
    Die hier bestimmte Kapazität $C_8=\qty{361(11)}{\nano\farad}$. 
    Die realen Werte sind $R_8'=\qty{564}{\ohm}$ und $C_8'=\qty{294,1}{\nano\farad}$.
    Die damit gebildeten Quotienten sind $\frac{R_8}{R_8'}=1,223\pm0,037$ und $\frac{C_8}{C_8'}=1,227\pm0,037$.
    Bei Wert 9 sind die gemessenen Werte $R_9=\qty{488(15)}{\ohm}$ und $C_9=\qty{511(7)}{\nano\farad}$.
    Für die tatsächlichen Werte gilt $R_9'=\qty{464,9}{\ohm}$ und $C_9=\qty{433,71}{\nano\farad}$.
    Damit sind die Verhältnisse $\frac{R_9}{R_9'}=1,05\pm0,03$ und $\frac{C_9}{C_9'}=1,178\pm0,016$.
    
    \subsection{Induktivitätsmessbrücke}
    Bei der Induktivitätsmessbrücke wurde mit Realwerten von $R_{18}'=\qty{360,5}{\ohm}$ und $L_{18}'=\qty{48,82}{\milli\henry}$ und den gemessenen Werten ebenfalls die Quotienten berechnet.
    Die eingesetzten gemessen Werte sind dabei $R_{18}=\qty{412(13)}{\ohm}$ und $L_{18}=\qty{34,16(1,04)}{\milli\henry}$.
    Es ergibt sich $\frac{R_{18}}{R_{18}'}=1,143\pm0,036$ und $\frac{L_{18}}{L_{18}'}=0,6997\pm0,0213$.

\label{sec:Diskussion}
