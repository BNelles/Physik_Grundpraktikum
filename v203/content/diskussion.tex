\section{Diskussion}
\label{sec:Diskussion}
\subsection{Bestimmung der Verdampfungswärme bei einem Druck von maximal 1 bar}
Die mit linearer Regression berechnete Verdampfungswärme beträgt $L=\qty{27.3(0.3)}{\kilo\joule\per\mole}$.
Nach \cite{LWasser} beträgt der Literaturwert bei $T=\qty{20}{\celsius}$ $L_{t,1}=\qty{44.2}{\kilo\joule\per\mole}$ $T=\qty{100}{\celsius}$ und bei $L_{t,2}=\qty{40.7}{\kilo\joule\per\mole}$.
Für die Abweichung gitl mit 
\begin{equation}
    |A|=\Bigl|\frac{L-L_t}{L_t}\Bigr|
    \label{eqn:abweichung}
\end{equation}
$\Bigl|\frac{L-L_{t,1}}{L_{t,1}}\Bigr|=\qty{38.23(0.007)}{\percent}$. 
Für $T=\qty{100}{\celsius}$ gilt $\Bigl|\frac{L-L_{t,1}}{L_{t,1}}\Bigr|=\qty{32.92(0.007)}{\percent}$.
Ein möglicher Grund für die Abweichung ist, dass L hier als konstant angenommen wurde.
Zudem könnte es sein, dass die Glaspparatur nicht hinreichend evakuiert wurde.  
Zudem sind auch Ablesefehler und Fehler durch die Messunsicherheit des Manometers möglich. 

\subsection{Bestimmung von der Arbeit um die Anziehungskräfte zwischen Molekülen zu überwinden}
Für den Wert der Arbeit, für die Überwindung der intermolekularen Kräfte, wurde $L_i=\qty{0.2501(0.0031)}{eV}$ berechnet.

\subsection{Bestimmung von $L$, wenn der Druck größer als $\qty{1}{\bar}$ ist.}
$L$ wird hier durch Gleichung \ref{eqn:L} beschrieben. Da $L$ gegen 0 geht, wenn sich das Wasser seinem kritischen Punkt
nähert, ist anzunehmen, dass nur die positive Lösung der Wurzel betrachtet werden muss, wie in Abbildung \ref{fig:LT_pos}
sichtbar wird. Jedoch sind die Fehler in dem Ausgleichspolynom sehr groß, wodurch der Verlauf von dem Graphen nachhaltig
beeinflusst werden könnte. Möglicherweise ist aber auch die Wahl eines Polynoms dritten Grades nicht gut.

