\section{Diskussion}
\label{sec:Diskussion}
\subsection{Bestimmung der Verdampfungswärme bei einem Druck von maximal 1 bar}
Die mit linearer Regression berechnete Verdampfungswärme beträgt $L=\qty{27.3(0.3)}{\kilo\joule\per\mole}$.
Nach \cite[LWasser] beträgt der Literaturwert bei $T=\qty{20}{\celsius}$ $L_{t,1}=\qty{44.2}{\kilo\joule\per\mole}$ $T=\qty{100}{\celsius}$ und bei $L_{t,2}=\qty{40.7}{\kilo\joule\per\mole}$.
Für die Abweichung gitl mit 
\begin{equation}
    |A|=\Bigl|\frac{L-L_t}{L_t}\Bigr|
    \label{eqn:abweichung}
\end{equation}
$\Bigl|\frac{L-L_{t,1}}{L_{t,1}}\Bigr|=\qty{38.23(0.007)}{\percent}$. 
Für $T=\qty{100}{\celsius}$ gilt $\Bigl|\frac{L-L_{t,1}}{L_{t,1}}\Bigr|=\qty{32.92(0.007)}{\percent}$.
Ein möglicher Grund für die Abweichung ist, dass L hier als konstant angenommen wurde.
Zudem könnte es sein, dass die Glaspparatur nicht hinreichend evakuiert wurde.  
Zudem sind auch Ablesefehler und Fehler durch die Messunsicherheit des Manometers möglich. 

\subsection{Bestimmung von der Arbeit um die Anziehungskräfte zwischen Molekülen zu überwinden}
Für den Wert der Arbeit, für die Überwindung der intermolekularen Kräfte, wurde $L_i=\qty{0.2501(0.0031)}{eV}$ berechnet.

