\section{Auswertung}
\label{sec:Auswertung}
\subsection{Bestimmung der Verdampfungswärme}
Die sowohl für $p \leq \qty{1}{\bar}$, als auch für $p>\qty{1}{\bar}$ aufgenommenen Werte für $T$ und $p$ wurden in der Tabelle \ref{tab:tabelle1} angeführt.
\begin{table}[htbp]
    \caption{In der Tabelle ist der Druck in Abhängigkeit zur Temperatur eingetragen.}
    \label{tab:tabelle1}
    \begin{minipage}[t]{0.3\linewidth}
    \begin{tblr}[t]{
        colspec={S[table-format=3.0] S[table-format=1.1] S[table-format=1.3] S[table-format=1.3]},
        row{1}={guard, mode=math},
        vline{2}={2}{-}{text=\clap{$\pm$}},
        vline{4}={2}{-}{text=\clap{$\pm$}},
    }
        \toprule
        \SetCell[c=2]{c} T \mathbin{/} \unit{\celsius} & &\SetCell[c=2]{c} p \mathbin{/} \unit{\bar}&\\
        \midrule
        19 & 0.5 &   0.057 & 0.001       \\
        20 & 0.5 &   0.076 & 0.001       \\
        21 & 0.5 &   0.081 & 0.001       \\
        22 & 0.5 &   0.086 & 0.001       \\
        23 & 0.5 &   0.090 & 0.001       \\
        24 & 0.5 &   0.094 & 0.001       \\
        25 & 0.5 &   0.098 & 0.001       \\
        26 & 0.5 &   0.102 & 0.001       \\
        27 & 0.5 &   0.105 & 0.001       \\
        28 & 0.5 &   0.109 & 0.001       \\
        29 & 0.5 &   0.113 & 0.001       \\
        30 & 0.5 &   0.116 & 0.001       \\
        31 & 0.5 &   0.120 & 0.001       \\
        32 & 0.5 &   0.124 & 0.001       \\
        33 & 0.5 &   0.127 & 0.001       \\
        34 & 0.5 &   0.131 & 0.001       \\
        35 & 0.5 &   0.135 & 0.001       \\
        36 & 0.5 &   0.138 & 0.001       \\
        37 & 0.5 &   0.143 & 0.001       \\
        38 & 0.5 &   0.146 & 0.001       \\
        39 & 0.5 &   0.150 & 0.001       \\
        40 & 0.5 &   0.154 & 0.001       \\
        41 & 0.5 &   0.157 & 0.001       \\
        42 & 0.5 &   0.161 & 0.001       \\
        43 & 0.5 &   0.165 & 0.001       \\
        44 & 0.5 &   0.170 & 0.001       \\
        45 & 0.5 &   0.175 & 0.001       \\
        46 & 0.5 &   0.179 & 0.001       \\
        47 & 0.5 &   0.184 & 0.001       \\
        48 & 0.5 &   0.189 & 0.001       \\
        49 & 0.5 &   0.194 & 0.001       \\
        50 & 0.5 &   0.197 & 0.001       \\
        51 & 0.5 &   0.204 & 0.001       \\
        52 & 0.5 &   0.210 & 0.001       \\
        53 & 0.5 &   0.215 & 0.001       \\
        54 & 0.5 &   0.221 & 0.001       \\
        \bottomrule 
    \end{tblr}
\end{minipage}
\hfill
\begin{minipage}[t]{0.3\linewidth}
    \begin{tblr}[t]{
        colspec={S[table-format=3.0] S[table-format=1.1] S[table-format=1.3] S[table-format=1.3]},
        row{1}={guard, mode=math},
        vline{2}={2}{-}{text=\clap{$\pm$}},
        vline{4}={2}{-}{text=\clap{$\pm$}},
    }
        \toprule
        \SetCell[c=2]{c} T \mathbin{/} \unit{\celsius} & &\SetCell[c=2]{c} p \mathbin{/} \unit{\bar}&\\
        \midrule
        55  & 0.5 &   0.226  & 0.001     \\
        56  & 0.5 &   0.231  & 0.001     \\
        57  & 0.5 &   0.238  & 0.001     \\
        58  & 0.5 &   0.246  & 0.001     \\
        59  & 0.5 &   0.251  & 0.001     \\
        60  & 0.5 &   0.257  & 0.001     \\
        61  & 0.5 &   0.263  & 0.001     \\
        62  & 0.5 &   0.268  & 0.001     \\
        63  & 0.5 &   0.278  & 0.001     \\
        64  & 0.5 &   0.285  & 0.001     \\
        65  & 0.5 &   0.293  & 0.001     \\
        66  & 0.5 &   0.301  & 0.001     \\
        67  & 0.5 &   0.311  & 0.001     \\
        68  & 0.5 &   0.320  & 0.001     \\
        69  & 0.5 &   0.328  & 0.001     \\
        70  & 0.5 &   0.340  & 0.001     \\
        71  & 0.5 &   0.349  & 0.001     \\
        72  & 0.5 &   0.361  & 0.001     \\
        73  & 0.5 &   0.373  & 0.001     \\
        74  & 0.5 &   0.386  & 0.001     \\
        75  & 0.5 &   0.400  & 0.001     \\
        76  & 0.5 &   0.416  & 0.001     \\
        77  & 0.5 &   0.434  & 0.001     \\
        78  & 0.5 &   0.450  & 0.001     \\
        79  & 0.5 &   0.469  & 0.001     \\
        80  & 0.5 &   0.488  & 0.001     \\
        81  & 0.5 &   0.505  & 0.001     \\
        82  & 0.5 &   0.524  & 0.001     \\
        83  & 0.5 &   0.545  & 0.001     \\
        84  & 0.5 &   0.564  & 0.001     \\
        85  & 0.5 &   0.587  & 0.001     \\
        86  & 0.5 &   0.605  & 0.001     \\
        87  & 0.5 &   0.627  & 0.001     \\
        88  & 0.5 &   0.641  & 0.001     \\
        89  & 0.5 &   0.656  & 0.001     \\
         \bottomrule 
    \end{tblr}
\end{minipage}
\hfill
\begin{minipage}[t]{0.3\linewidth}
    \begin{tblr}[t]{
        colspec={S[table-format=3.0] S[table-format=1.1] S[table-format=1.3] S[table-format=1.3]},
        row{1}={guard, mode=math},
        vline{2}={2}{-}{text=\clap{$\pm$}},
        vline{4}={2}{-}{text=\clap{$\pm$}},
    }
        \toprule
        \SetCell[c=2]{c} T \mathbin{/} \unit{\celsius} & &\SetCell[c=2]{c} p \mathbin{/} \unit{\bar}&\\
        \midrule
    90  & 0.5 &   0.676  & 0.001     \\
    91  & 0.5 &   0.701    & 0.001   \\
    92  & 0.5 &   0.734    & 0.001   \\
    93  & 0.5 &   0.759    & 0.001   \\
    94  & 0.5 &   0.782    & 0.001   \\
    95  & 0.5 &   0.806    & 0.001   \\
    96  & 0.5 &   0.832    & 0.001   \\
    97  & 0.5 &   0.861    & 0.001   \\
    98  & 0.5 &   0.890    & 0.001   \\
    99  & 0.5 &   0.914    & 0.001   \\
    100 & 0.5 &   0.947    & 0.001   \\
    101 & 0.5 &   0.955    & 0.001   \\
    102 & 0.5 &   0.958    & 0.001   \\
    103 & 0.5 &   0.962    & 0.001   \\
    104 & 0.5 &   0.968    & 0.001   \\
    105 & 0.5 &   0.970    & 0.001   \\
    109 & 0.5 &   0.971    & 0.001   \\
    110 & 0.5 &   0.971    & 0.001   \\
    111 & 0.5 &   0.974    & 0.001   \\
    112 & 0.5 &   0.974    & 0.001   \\
    113 & 0.5 &   0.974    & 0.001   \\
    118 & 0.5 &   1        & 0.001   \\
    131 & 0.5 &   2        & 0.001   \\
    141 & 0.5 &   3        & 0.001   \\
    149 & 0.5 &   4        & 0.001   \\
    156 & 0.5 &   5        & 0.001   \\
    161 & 0.5 &   6        & 0.001   \\
    167 & 0.5 &   7        & 0.001   \\
    172 & 0.5 &   8        & 0.001   \\
    173 & 0.5 &   9        & 0.001   \\
    181 & 0.5 &  10        & 0.001   \\
    186 & 0.5 &  11        & 0.001   \\
    189 & 0.5 &  12        & 0.001   \\
    192 & 0.5 &  13        & 0.001   \\
    195 & 0.5 &  14        & 0.001   \\
    198 & 0.5 &  15        & 0.001   \\
    \bottomrule 
    \end{tblr}
    \end{minipage}
    \hfill
\end{table}

Um einen Mittelwert für die Verdampfungswärme zu bestimmen, wurde die Temperatur in Kelvin umgerechnet und der Kehrwert gebildet.
Für den Druck wurde eine logarithmische Skala gewählt.
Diese Daten wurden in der Abbildung \ref{fig:werte} dargestellt.

\begin{figure}[H]
    \centering
    \includegraphics{plot1.pdf}
    \caption{Hier ist der Dampfdruck in Bar auf einer logarithmischen Skala gegen die Temperatur in Kelvin aufgetragen.}
    \label{fig:werte}
\end{figure}

Für den Bereich $p \leq \qty{1}{\bar}$ wurde dann lineare Regression angewandt.
Die in der Abbildung \ref{fig:regression} eingezeichnete Ausgleichsgerade besitzt die Steigung $a=\qty{-3282.759(36.723)}{\kelvin}$.  
Der y-Achsenabschnitt mit $b=\qty{8.6(0.11)}{}$ ist im Gegensatz zu a vernachlässigbar klein.
Diese entspricht $T\.\ln(\frac{p}{p_0})$.
Gleichung \ref{eqn:druck} wird nach der Verdampfungswärme umgestellt und die Steigung der Ausgleichsgeraden wird in 
\begin{equation*}
    L=-R\cdot T\.\ln(\frac{p}{p_0})
\end{equation*}
eingesetzt.
Mit $R=\qty{8.3145}{\joule\per\mole\kelvin}$
Damit ergibt sich $L=\qty{27.3(0.3)e-3}{\joule\per\mole}$
\begin{figure}[H]
    \centering
    \includegraphics{plot.pdf}
    \caption{Dargestellt ist der Dampfdruck in Bar in logarithmischer Skala, in Abhängigkeit zur Temperatur in Kelvin.
    Zudem wurde eine Ausgleichsgerade eingefügt.}
    \label{fig:regression}
  \end{figure}

\subsection{Bestimmung von der Arbeit um die Anziehungskräfte zwischen Molekülen zu überwinden}
Die Arbeit $L_i$ die benötigt wird, um die intermolekularen Anziehungskräfte zu überwinden, wird mit $L_i=L-L_a$ berechnet.
$L_a$ wird mit $T_a=\qty{373}{\kelvin}$ ausgerechnet.
Es emtspricht $L_a=pV$ und kann in Formel \ref{eqn:gas} eingesetzt werden.
Demnach ist $L_a=RT$ und das entspricht $L_a=\qty{3.1e-3}{\joule\per\mole}$.
Daraus folgt $L_i=\qty{24.1(0.3)e-3}{\joule\per\mole}$.
Um die Arbeit für ein einzelnes Molekül zu erhalten, wird $L_I$ durch die Avogradokonstante $N_A=\qty{6.022e23}{\per\mole}$ dividiert.
Umgerechnet in Elektronenvolt ergibt sich $L_i=\qty{0.2501(0.0031)}{eV}$.   

\subsection{Bestimmung der Verdampfungswärme wenn der Druck größer als $\qty{1}{\bar}$ ist.}

Hier wird zunächst nicht mehr davon ausgegangen, dass in der Clausius-Clapeyron Gleichung \ref{eqn:clausius} $L$ konstant ist
und von $V_D$ der allgemeinen Gasgleichung beschrieben wird. Stattdessen gilt hier, dass
\begin{equation}
    \left(p+\frac{\symup{a}}{V^2}\right)=\symup{R}T
    \label{eqn:vd}
\end{equation}
ist. Allerdings ist $V_F$ immernoch vernachlässigbar gegenüber $V_D$.\\
Um nun $L$ zu bestimmen wird zunächst die Clausius-Clapexronische Gleichung umgestellt zu
\begin{equation}
    -TV_D\frac{\symup{d}p}{\symup{d}T}=L \text{ .}
    \label{eqn:claus}
\end{equation}

Zur Bestimmung von $p(T)$ werden die Werte aus \ref{tab:tabelle1} für $p>\qty{1}{bar}$ betrachtet und mit einer 
Regressionspolynom dritten Grades beschrieben\ref{fig:druck2} $p(T)=AT^3+BT^2+CT+D$.

\begin{figure}[H]
    \centering
    \includegraphics{plot2.pdf}
    \label{fig:druck2}
\end{figure}

Hieraus ergeben sich die Werte für die Koeffizienten als
\begin{itemize}
    \item $A=\qty{4.70(6.12)e-6}{\newton\per\meter\squared\kelvin\cubed}$
    \item $B=\qty{-4.54(7.93)e-3}{\newton\per\meter\squared\kelvin\squared}$
    \item $C=\qty{1.46(3.42)}{\newton\per\meter\squared\kelvin}$
    \item $D=\qty{-1.57(4.91)e2}{\newton\per\meter\squared}$
\end{itemize}

Zur bestimmung von $V_D$ in Abhängigkeit von der Temperatur wird als erstes Gleichung \ref{eqn:vd} danach umgeformt
woraus man durch umformen die Gleichung 
\begin{equation}
    V_{D}^2-\frac{\symup{R}T}{p}V_D+\frac{a}{p}=0 \implies 
    V_D=\frac{1}{p}\left(\frac{1}{2}\symup{R}T\pm\sqrt{\left(\frac{1}{2}\symup{R}T\right)^2-ap}\right)
    \label{eqn:vol}
\end{equation}
bekommt.\\
Wenn nun die Gleichungen \ref{eqn:vol} und \ref{eqn:druck2} in die Clausius-Clapexronische Gleichung \ref{eqn:claus} eingesetzt
werden, ist ein ausdruck von $L(T)$ nur in Abhängigkeitvon $T$ gefunden.
\begin{align}
    &L=-\frac{T}{AT^3+BT^2+CT+D}\left(\frac{\symup{R}T}{2}\pm\sqrt{\left(\frac{1}{2}RT\right)^2-a(AT^3+BT^2+CT+D)}\right)(3AT^2+2BT+C)\\
    \iff &L=-\frac{3AT^3+2BT^2+CT}{AT^3+BT^2+CT+D}\left(\frac{\symup{R}T}{2}\pm\sqrt{\left(\frac{1}{2}RT\right)^2-a(AT^3+BT^2+CT+D)}\right)
    \label{eqn:L}
\end{align}

\begin{figure}[H]
    \centering
    \includegraphics[width=\textwidth]{plot3.pdf}
    \label{fig:LT_pos}
    \caption{L gegen T, wenn die Wurzel positivi ist.}
\end{figure}
\begin{figure}[H]
    \centering
    \includegraphics[width=\textwidth]{plot4.pdf}
    \caption{L gegen T, wenn die Wurzel negativ ist.}
\end{figure}
