\subsection{Messung der Durchmesser der Löcher}

Zunächst wurde die Schallgeschwindigkeit in Acryl über das Zeitintervall $\increment t_1$ gemessen, welches den zeitlichen
Abstand zwischen der Aussendung eines Impulses und dessen Empfang wiedergibt. Dabei wurde die Weglänge zwischen der 
Oberseite des Acrylblocks und der Oberseite der Löcher $x_{\symup{A}}$ mithilfe einer Schieblehre ausgemessen.
\begin{figure}[H]
    \includegraphics{build/schall.pdf}
    \centering
    \caption{Zeit-Strecken Diagramm für die Schallwellen im Acrylblock.}
    \label{fig:schall}
\end{figure}
\noindent Dabei folgt der Graph der Gleichung
\begin{equation}
    x_{\symup{A}}=\symup{c}_{\symup{A}} \frac{\increment t_1}{2}-x_{\symup{M}}
    \label{eqn:gerade}
\end{equation}
wobei $x_{\symup{M}}$ die halbe Distanz ist, die die Schallwelle in dem Messgerät selbst zurücklegt, da die Strecke zwei
Mal durchlaufen wird.
Aus Abbildung \ref{fig:schall} folgt dann für, dass $\symup{c}_{\symup{A}}=\qty{2862(91)}{\meter\per\second}$ und
$x_{\symup{M}}=\qty{5.2(1.3)}{\milli\meter}$ sind.\\
Zur Bestimmung der Durchmesser der Löcher wird dann die Zeit $\increment t_2$ vermessen, welche den Abstand zwischen
der Gegenüberliegenden Oberfläche zu der von $\increment t_1$ und den Löchern darstellt.
Die Durchmesser sind dann mit der Höhe des Blocks $\symup{h}$ über
\begin{equation}
    d=\symup{h}-x_{\symup{A}}-\symup{c} \increment t_2
    \label{eqn:durchmesser}
\end{equation}
bestimmt.
\begin{table}
\captionsetup[table]{position=bottom} 
\centering
\label{tab:durchmesser}
\sisetup{table-format=1.1, per-mode=reciprocal}
\begin{tblr}{
colspec = {S[table-format=2.1] S[table-format=1.1] S[table-format=1.1] S[table-format=3.2] S[table-format=3.2]},
  row{1} = {guard, mode=math},
  vline{2}={1}{-}{text=\clap{$\pm$}},
  vline{5}={4}{-}{text=\clap{$\pm$}},
}
\toprule
x_{\symup{A} 1} \text{/} \unit{\milli\meter} &   &x_{\symup{A} 1_{\symup{t}}} \text{/} \unit{\milli\meter} &\text{Fehler} \text{/} \unit{\percent} & \\
\midrule
53  & 2  &  52.5  &   1 &  4 \\
 9  & 1  &   4.1  & 117 & 33 \\
10  & 1  &  12.1  &  14 & 11 \\
19  & 2  &  20.0  &   5 &  7 \\
27  & 2  &  28.0  &   2 &  6 \\
36  & 2  &  37.0  &   3 &  5 \\
44  & 2  &  43.7  & 0.3 &  5 \\
52  & 2  &  51.3  &   1 &  4 \\
59  & 2  &  58.7  &   1 &  4 \\
15  & 1  &  15.6  &   7 &  9 \\
17  & 2  &  17.2  &   4 &  8 \\
\bottomrule
\end{tblr}
\caption{Die berechneten Distanzen zu den Löchern $x_{\text{A}1}$ mit ihren entsprechenden theoretischen Werten $x_{\symup{A}1_{\symup{t}}}$ 
und der Abweichung zwischen ihnen.}
\end{table}

\begin{table}[H]
    \captionsetup[table]{position=bottom} 
      \centering
      \label{tab:durchmesser}
      \sisetup{table-format=1.1, per-mode=reciprocal}
    \begin{tblr}{
      colspec = {S[table-format=2.0] S[table-format=1.0] S[table-format=2.1] S[table-format=2.3] S[table-format=1.3]},
        row{1} = {guard, mode=math},
        vline{2}={1}{-}{text=\clap{$\pm$}},
        vline{5}={4}{-}{text=\clap{$\pm$}},
    }
    \toprule
    x_{\symup{A} 2} \text{/} \unit{\milli\meter} &   &x_{\symup{A} 2_{\symup{t}}} \text{/} \unit{\milli\meter} &\text{Fehler} \text{/} \unit{\percent} & \\
    \midrule
    12  & 1  &  19.8  & 38 & 7 \\
    70  & 3  &  74.9  &  6 & 4 \\
    62  & 3  &  66.9  &  8 & 4 \\
    53  & 2  &  59.1  & 10 & 4 \\
    45  & 2  &  50.9  & 12 & 4 \\
    37  & 2  &  42.0  & 12 & 4 \\
    28  & 2  &  34.5  & 20 & 5 \\
    19  & 1  &  25.8  & 28 & 6 \\
    10  & 1  &  17.3  & 44 & 8 \\
    60  & 2  &  63.9  &  7 & 4 \\
    58  & 2  &  61.9  &  6 & 4 \\
    \bottomrule
    \end{tblr}
    \caption{Die berechneten Distanzen zu den Löchern $x_{\symup{A}2}$ mit ihren entsprechenden theoretischen Werten $x_{\symup{A}2_{\symup{t}}}$ 
  und der Abweichung zwischen ihnen.}
\end{table}


\begin{table}[H]
  \captionsetup[table]{position=bottom} 
    \centering
    \label{tab:durchmesser}
    \sisetup{table-format=1.1, per-mode=reciprocal}
  \begin{tblr}{
    colspec = {S[table-format=2.1] S[table-format=1.1] S[table-format=1.1] S[table-format=3.2] S[table-format=3.2]},
      row{1} = {guard, mode=math},
      vline{2}={1}{-}{text=\clap{$\pm$}},
      vline{5}={4}{-}{text=\clap{$\pm$}},
  }
  \toprule
  d \text{/} \unit{\milli\meter} &   &d_{\symup{t}} \text{/} \unit{\milli\meter} &A \text{/} \unit{\percent} & \\
  \midrule
  14.9  & 3.5  &    7.7 &    93.20  &    45.36\\
   1.0  & 3.8  &    1.0 &     0.97  &   380.66\\
   7.9  & 3.7  &    1.0 &   692.13  &   364.68\\
   7.8  & 3.7  &    0.9 &   769.03  &   405.46\\
   7.9  & 3.7  &    1.1 &   620.13  &   331.53\\
   7.3  & 3.7  &    1.0 &   633.48  &   366.02\\
   8.8  & 3.6  &    1.8 &   387.78  &   201.52\\
   9.7  & 3.6  &    2.9 &   232.86  &   124.41\\
  11.1  & 3.6  &    4.0 &   177.82  &    89.38\\
   5.8  & 3.7  &    0.5 &  1057.85  &   739.39\\
   5.5  & 3.7  &    0.9 &   509.86  &   411.38\\
   \bottomrule
  \end{tblr}
  \caption{Die berechneten Durchmesser der Löcher $d$ mit ihren entsprechenden theoretischen Werten $d_{\symup{t}}$ 
und der Abweichung $A$ zwischen ihnen}
\end{table}

