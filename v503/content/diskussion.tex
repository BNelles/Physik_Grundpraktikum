\section{Diskussion}

Abweichungen von theoretischen Werten werden hier mittels der Formel
\begin{equation}
    \delta w=\frac{w_T-w_E}{w_T}*100
\end{equation}
berechnet, wobei $w_T$ der theoretische Wert ist und $w_E$ ein experimentell bestimmter Wert.\\
\noindent Die berechnete Elementarladung $e=\qty{1.54(0.37)e-19}{\coulomb}$ weicht von der theoretischen Elementarladung
$e_0=\qty{1.6e-19}{\coulomb}$ um $\qty{4(23)}{\percent}$ ab. Bei dem Versuch wurde die Messreihe des zweiten 
Tröpfchens sehr schnell verworfen, da dessen durchschnittliche Steiggeschwindigkeit größer war, als dessen 
Fallgeschwindigkeit, was dazu führte, dass der zu berechnende Radius komplex wurde. Desweiteren wurde die berechnete
Ladung des ersten Tröpfchens aufgrund seiner hohen sehr Standartabweichung aus der finalen Auswertung ausgeschlossen.
Erklären lassen sich die beiden Fehler mit der hohen Geschwindigkeit, mit der sich die Tröpfchen am Gitter bewegten
und es somit zu großen Fehlern in der Zeitmessung kam durch Faktoren, wie eine unzuverlässige Reaktionszeit.
Alternativ spielt auch eine schwierige Koordination zwischen der Umpolung der Spannung und der Zeitmessung 
dann eine große Rolle.\\
\noindent Der berechnete Wert der Avogadro-Konstante $N_A=\qty{6.3(1.5)e23}{\per\mole}$ hat eine Abweichung von
$\qty{4(25)}{\percent}$ von dem Theoriewert $N_{a_t}=\qty{6.02}{\per\mole}$ \cite{Avo}.
\label{sec:Diskussion}
