\section{Theorie}
\label{sec:Theorie}

Es wird für diesen Versuch die Annahme getroffen, dass die Eigenschaft der Ladung quantisiert ist.
Beim Öltröpfchenversuch werden Öltröpfchen zwischen zwei Kondensatorplatten gebracht und damit ein elektrisches Feld angelegt.
Die Öltröpfchen erfahren die Kraft
\begin{equation}
    \vec{F}_g=m \vec{g}
    \label{eqn:Schwerefeld}
\end{equation}
durch das Schwerefeld der Erde, welche stets nach unten gerichtet ist.
Durch die Luft, in der sich das Tröpfchen bewegt, wirkt auch eine Reibungskraft.
Diese wird hier mit der Formel der Stokes'schen Reibung 
\begin{equation}
    \vec{F}_R=6 \pi r \eta_L \vec{v}
    \label{eqn:Stokes}
\end{equation}
berechnet.
Die Richtung der Stokes'schen Reibung ist immmer entgegengesetzt zu der Bewegungsrichtung des Tröpfchens.
Das elektrische Feld übt auf diese die Kraft
\begin{equation}
    \vec{F}_{el}=q \vec{E}
    \label{eqn:Efeld}
\end{equation}
aus.
Bei dieser Kraft hängt die Richtung von der Ladung der Platte und der Ladung des Tröpfchens ab.
Hat die Ladung der unteren Platte ein anderes Vorzeichen als die Ladung des Öltröpchens, so wirkt $\vec{F}_{el}$, wie in \ref{fig:Kraft} abgebildet, nach unten.
Damit bewegt sich das Tröpfchen nach unten und die Reibungskraft wirkt nach oben.
Die Reibung steigt so lange mit der Geschwinidgketi an, bis sich ein Kräftegleichgewicht einstellt.
Das Kräftegleichgewicht ist hier 
\begin{equation*}
    F_g+F_{el}=F_R \. .
\end{equation*}
Wenn die Vorzeichen der Ladungen von der unteren Platte und von dem Tröpfchen gleich sind, und $F_{el}>F_g$ gilt, so bewegt es sich nach oben, womit sich auch die Richtung der Stokes'schen Reibung umkehrt.
Dann ist 
\begin{equation*}
    F_g+F_R=F_{el}
\end{equation*}
das Kräftegleichgewicht.

\begin{figure}[H]
    \includegraphics{Bilder/öl.png}
    \centering
    \caption{Abgebildet ist ein Schema des Öltröpfchenversuches mit eingezeichneten Kräften \cite{V503}.}
    \label{fig:Kraft}
\end{figure}

\noindent Aus den beiden Kräftegleichgewichten kann die Formel
\begin{equation*}
    r=\sqrt{\frac{9 \eta_L (v_{ab}-v_{auf})}{4g(\rho_{Oel}-\rho_L)}}
\end{equation*}
hergeleitet werden.
Durch das $\rho_L$ wird hier der Auftrieb mit einberechnet.
Da aber
\begin{equation*}
    \rho_{Oel}>>\rho_L
\end{equation*}
gilt, ist der Auftrieb vernachlässigbar und die Formel vereinfacht sich zu 
\begin{equation*}
    r=\sqrt{\frac{9 \eta_L (v_{ab}-v_{auf})}{4g(\rho_{Oel})}} \. .
\end{equation*}

Die Formel für die Stokes'sche Reibung bringt aber Einschränkungen mit sich.
Für Öltröpchen, deren Durchmesser kleiner als die mittlere freie Weglänge von Luft sind, muss die Viskosität von Luft durch den Cunningham-Korrekturterm
\begin{equation}
    \eta_{eff}=\eta_L \Biggr(  \Biggl)
\end{equation}




\cite{V503}
