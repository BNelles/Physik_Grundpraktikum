\section{Durchführung}
Bei dem Versuch nimmt eine Kamera Öltropfchen auf, die an einem Gitter, welches Abstände definiert, herunterfließen.
Dabei stehen die Tröpfchen unter dem Einfluss der Gravitation, welche sie immer nach oben zieht,
da das Bild der Kamera invertiert ist, und einem elektrischen Feld, welches nach belieben umgepolt werden kann. 
Des weiteren erfahren die Tröpfchen Reibung am Gitter. Von oben werden dann Öltröpfchen in die Apparatur 
hineingesprüht, zwischen zwei Kondensatorplatten mit einem Abstand von $\qty{7.6250(0.0051)}{\milli\meter}$. 
Die angelegte Spannung beträgt hierbei $\qty{249}{\volt}$. Durch die Umpolung des entstehenden elektrischen
Feldes kann dann die Bewegungsrichtung veräändert werden. Danach wird eine Distanz festgelegt, die das betrachtete
Teilchen zurülegen soll zwischen umpolungen und die Zeit, die das Tröpfchen dann braucht, gemessen. Für jedes Tröpfchen
wird dies mehrmals wiederholt.
\label{sec:Durchführung}
