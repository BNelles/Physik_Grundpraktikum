\section{Durchführung}
\label{sec:Durchführung}

\noindent Zunächst wird die Zählrate einer $\beta^-$-Strahlungsquelle gemessen, wobei die Spannung in $\qty{20}{\volt}$ Schritten
variiert wird und dann die Anzahl der detektierten Teilchen nach $\qty{1}{\minute}$ aufgenommen. Dies wird solange 
geamcht, bis es zu einem signifikanten Anstieg der gemessenen Teilchen über mehrere Messungen hinweg kommt.
Danach wird die Totzeit zunächst an einem Oszilloskop abgelesen.\\

\noindent Daraufhin wird die Zählrate der Quelle nocheinmal über einen Zeitraum von $\qty{2}{\minute}$ aufgenommen.
zusätzlich dazu wird die Zählrate einer zweiten Quelle gemessen und deren kombinierte Zählrate.