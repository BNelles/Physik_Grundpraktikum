\section{Diskussion}

Die geringe Steigung $s=\qty{1.9904(0.3183)e-04}{\percent\per\volt}$ der Ausgleichsgerade lässt darauf schließen, dass 
der Effekt der Nachentladungen in dem  Zählrohr gering ist und dieses eine gute Güte hat. Jedoch ist die tatsächliche 
größe der Totzeit unklar, da der über das Oszilloskop bestimmte Wert $\tau_1=\qty{250(25)}{\micro\second}$ 
$\frac{\tau_1}{\tau_2}=2.5$ Mal größer  ist, als der über die Zählrate bestimmte Wert $\tau_2=\qty{96.508}{\second}$.
Diese diskrepanz kommt wahrscheinlich daher, dass die Totzeit am Oszilloskop nur schlecht ablesbar war.
Desweiteren wurden zu wenig Ereignisse bei der Messung der Zählrate bestimmt da die Anzeige zwischen $67000 \text{und }
68000$ wieder auf $1$ zurückgesprungen ist, der genaue Wert, wann dies passiert, aber unklar ist. Deswegen wurde er auf
einen Wert von $67000$ festgelegt, für den Zeitpunkt des Sprungs.
Daher lässt sich nicht gut einschätzen wie schnell sich die Ionenbarriere um die Anode herum auflöst. \\

\noindent Ein längerer aufenthalt dieser Ionen könnte aber möglicherweise die die geringe Anzahl an Sekundärelektronen 
erklären, da es möglich ist, dass diese von noch übrigen Ionen aufgefangen werden können, was diese ebenfalls 
neutralisiert und noch mehr Sekundärelektronen verhindert.

\label{sec:Diskussion}
