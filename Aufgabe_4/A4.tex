\input{header.tex}




\begin{document}





    \section{Theorie}
    	Eine Feder mit Federkonstante $\symbf{D}$ ist senkrecht zum Boden an einem Kraftmessgerät aufgehängt.
        Die Feder ist auf der anderen Seite mit einem Seil verbunden, dass über eine feste Rolle um 90° gedreht wird.
        Entlang dem Seil ist ein Lineal angebracht. Das Ende des Seils, wenn dieses locker hängt sollte sich dabei bei der Position 0$\unit{cm}$ befinden.

    \section{Durchführung}
        Das Seil wird um eine gewisse Entfernung entlang des Lineals ausgelenkt.
        Dann wird die Auslenkung vom Ruhezustand am Lineal abgelesen.
        Zusätzlich soll die Kraft am Kraftmessgerät abgelesen werden.
        Danach berechnet man die Federkonstante mit 
        \begin{equation}
            \symbf{D}=\frac{\symbf{F}}{\increment\symbf{x}}
        \end{equation}
    \section{Auswertung}
    \begin{table}
        \centering
        \caption{Messwerte}
        \sisetup{table-format=1.2}
        \begin{tblr}{
            colspec={S S S[table-format=1.0]},
            row{1}={guard, mode=math},
            }
            \toprule
            $\increment$x/\unit{\meter} & F/\unit{\newton} & D/\unit[per-mode=fraction]{\newton\per\meter} \\
            \midrule    
            0.05 & 0.15 & 3   \\
            0.1  & 0.29 & 2.9  \\
            0.15 & 0.44 & 2.93 \\
            0.2  & 0.59 & 2.95 \\
            0.25 & 0.74 & 2.96 \\
            0.3  & 0.89 & 2.97 \\
            0.35 & 1.04 & 2.97 \\
            0.4  & 1.19 & 2.98 \\
            0.45 & 1.34 & 2.98 \\
            0.5  & 1.49 & 2.98 \\
            \bottomrule
        \end{tblr}
    \end{table}
    Der Mittelwert der gemessenen Federkonstanten beträgt dabei 2,962$\unit[per-mode=fraction]{\newton\per\meter}$.\\
    Die Steigung der Regressionsgeraden beträgt 2,99$\unit[per-mode=fraction]{\newton\per\meter}$.
\begin{figure}
    \centering
    \includegraphics[width=\textwidth]{plot.pdf}
    \caption{Lineare Regression}
\end{figure}

\end{document}

