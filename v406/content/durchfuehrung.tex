\section{Durchführung}
Zum Messen des Interferenzmusters wird ein Laser auf eine Blende gerichtet, hinter der
eine Messvorrichtung platziert ist. Diese ist mit einem Strommessgerät verbunden, welches proportional zur Intensität
ausschlägt. Bevor der Laser eingeschltet wird, muss der Dunkelstrom gemessen werden, um äußere Störungen zu bestimmen.
Es wird in einem Intervall von $\qty{0}{\milli\meter}$ bis $\qty{50}{\milli\meter}$ vermessen, wobei das 
Intensitätsmaximum ungefähr mittig liegt. Dafür wird zwischen $\qty{20}{\milli\meter}$ und $\qty{30}{\milli\meter}$ in
Schritten von $\qty{0.5}{\milli\meter}$ vermessen und sonst in Schritten von $\qty{1}{\milli\meter}$. Dies wird für 
drei Blenden jeweils durchgeführt. Der Laser strahlt hier Licht mit einer Wellenlänge von $\qty{635}{\nano\meter}$ ab, 
Der Abstand zwischen dem Laser und der Blende beträgt $\qty{30}{\centi\meter}$ und der Abstand zwischen der Blende und dem
Laser beträgt $\qty{100}{\centi\meter}$.
\label{sec:Durchführung}
