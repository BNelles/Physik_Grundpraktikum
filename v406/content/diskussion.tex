\section{Diskussion}

Zur Berechnung der Abweichungen von den theoretischen Werten $w_t$ wurde die Gleichung 
\begin{equation}
    \frac{w_t-w_e}{w_t}\cdot \qty{100}{\percent}
    \label{eqn:ABweichung}
\end{equation}
\noindent verwendet, 
wobei $w_e$ der experimentell bestimmte Wert ist. Für den ersten Einzelspalt, mit dem Theoriewert $b_{1_t}\qty{0.075}{\milli\meter}$
und dem experimentellen Wert $b_1=\qty{0.0294(0.0020)}{\milli\meter}$, ergibt sich die Abweichung von $A_1\qty{60.8(2.7)}{\percent}$ vom Theoriewert. 
Für den zweiten Einzelspalt mit einem Theoriewert von $b_{1_t}\qty{0.15}{\milli\meter}$ und der experimentell bestimmten Spaltbreite $b_2=\qty{0.0901(0.0035)}{\milli\meter}$, folgt eine Abweichung von $\qty{39.9(2.3)}{\percent}$.
Die Spaltbreite des Doppelspaltes, gemessen $b_{DS}=\qty{0.059(0.008)}{\milli\meter}$ und der Abstand $s_{DS}=\qty{0.402(0.006)}{\milli\meter}$, 
besitzt nach Herstellerangaben die Werte $b_{DS_t}=\qty{0.1}{\milli\meter}$ und der Abstand $s_{DS}=\qty{0.4}{\milli\meter}$
Die Abweichung beträgt für die Spaltbreite $\qty{41(8)}{\percent}$ und der Spaltabstand eine von $\qty{0.5(1.5)}{\percent}$.
Die relativ großen Fehler können durch mögliche Fehler bei der Aufstellung der Ausgleichsfunktion, sowie dadurch erklärt werden, dass das Beugungsbild nur durch Augenmaß ausgerichtet
wurde und sich das größte Intensitätsmaximum dadurch möglicherweise nicht genau bei $x=\qty{25}{\centi\meter}$ befindet.

\label{sec:Diskussion}
