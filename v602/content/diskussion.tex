\section{Diskussion}
\label{sec:Diskussion}

\subsection{Bestimmung des Absorbtionsspektrums einer CU-Röntgenröhre}
Der berechnete Braggwinkel beträgt $\theta=14°$, während der experimentell bestimmte einen Wert vom $\theta_e=13.8°$ besitzt.
Demnach ist die Abweichung, bestimmt mit \ref{eqn:Abweichung} $A_\theta=\qty{1.43}{\percent}$.

\begin{equation}
    A_{et}=\Biggl|\frac{w_e-w_t}{w_t}\Biggr|
    \label{eqn:Abweichung}
\end{equation}

\noindent
Die Energie der $\text{K}_{\alpha}$-Linie von Kupfer beträgt $E_{\alpha}=\qty{8.08(0.07)}{\kilo\electronvolt}$ und
 die der $\text{K}_{\beta}$-Linie $E_{\beta}=\qty{8.91(0.08)}{\kilo\electronvolt}$.

\subsection{Bestimmung der Absorbtionsspektren verschiedener Materialien}
Die $\text{K}_{\alpha}$-Linie von Kupfer wurde als $E_{\symup{b}}=\qty{13.68}{\kilo\electronvolt}$ gemessenen.
Bei Gallium ergibt sich $E_{\symup{g}}=\qty{10.38}{\kilo\electronvolt}$ und bei Strontium $E_{\symup{s}}=\qty{16.13}{\kilo\electronvolt}$.
Zudem sind bei Zink und Zirkonium $E_{\symup{z}}=\qty{9.70}{\kilo\electronvolt}$ und $E_{\symup{zr}}=\qty{18.18}{\kilo\electronvolt}$.

\noindent Die Literaturwerte sind in Tabelle \ref{tab:absorb} angegeben.
Die mit \ref{eqn:Abweichung} berechneten Abweichungen sind dann

\begin{table}[H]
    \centering
    \caption{Aufgelistet sind die Abweichungen von den experimentellen und theoretischen Werten der $\text{K}_{\alpha}$-Linie von mehereren Elementen.}
    \label{tab:Abweichung}
    \sisetup{table-format=1.1, per-mode=reciprocal}
    \begin{tblr}{
      colspec = {S[table-format=2.0] S[table-format=1.2]},
        row{1} = {guard, mode=math},
    }
    \toprule
    Z  &   A \text{/} \unit{\percent}\\
    \midrule
    35  & 1.56 \\
    31  & 0.10 \\
    38  & 0.19 \\
    30  & 0.52  \\
    40  & 1.06 \\
    \bottomrule
    \end{tblr}
  \end{table}

Damit ist die größte Abweichung hier $A_G=\qty{1.56}{\percent}$.

\subsection{Bestimmung der Rydbergenergie}
Die hier bestimmte Rydbergenergie beträgt $R_{\infty}=\qty{14.3(0.1)}{\electronvolt}$.
Der Literaturwert ist dabei $R_{\infty}=\qty{13.6}{\electronvolt}$.
Die Abweichung beträgt mit Formel \ref{eqn:Abweichung} $A_{R_{\infty}}=\qty{4.954(0.734)}{\percent}$.

\noindent Generell kann man sagen, dass die Abweichung insgesamt relativ gering ausfallen.
Dies kann damit bebründet werden, dass die Aparatur gut funktioniert und kaum äußere Einflüsse die Messung verfälschen.