\input{header.tex}

\subject{v101}
\title{Das Trägheitsmoment}
\date{%
  Durchführung: 14.11.2023
  \hspace{3em}
  Abgabe: 21.11.2023
}

\begin{document}

\maketitle
\thispagestyle{empty}
\tableofcontents
\newpage

\section{Theorie}
\label{sec:Theorie}
    \subsection{allgemeine Formeln}
    Das ohmsche Gesetz beschreibt den Zusammenhang zwichen der Spannung an einem Leiter mit der Stromstärke.
    \begin{equation}
        U=R \cdot I
        \label{eqn:ohm}
    \end{equation}\\
    Die kirchhoffschen Gesetze beschreiben das Verhalten von Strom in einem geschlossenen Stromkreis.
    Die Knotenregel besagt, dass alle einem Knoten hinzugefügten Ladungen gleich der abgegebenen Ladungen sein müssen.
    \begin{equation}
        \sum_{i=1}^{\symup{n}}{I_i}=0
        \label{eqn:knoten}
    \end{equation}\\
    Die Maschenregel besagt, dass die Summe aller einzelnen Spannungen in einer Masche gleich 0 ist. 
    Das liegt daran, dass die zugeführte und abgegebene elekztrische Arbeit gleich groß sein muss.
    \begin{equation}
        \sum_{i=1}^{\symup{n}}{U_{0,i}} - \sum_{j=1}^{\symup{m}}{U_{ab,j}} =0
        \label{eqn:masche}
    \end{equation}

    \subsection{Wheatstonsche Brückenschaltung}
    Die Wheatstonsche Brücke ist eine Schaltung, die genutzt wird, um ohmsche Widerstände zu messen.
    Dabei ist eine Spannung an zwei parallel geschalteten Paaren aus Widerständen angeschlossen, wobei nach dem ersten Widerstand jeweils ein Kabel an einem Oszilloskop angeschlossen ist.
    \ref{fig:Wheatstone}
    Drei der Widerstände sind bekannt, der andere ist mit der Formel
    \begin{equation}
        R_x=R_2 \frac{R_3}{R_4}
        \label{eqn:widerstand}
    \end{equation}
    zu bestimmen.

    \subsection{Kapazitätsmessbrücke}
    Hier werden, im Gegensatz zu der Wheatstonsche Brückenschaltung, vor die beiden Widerstände auf der linken Seite jeweils ein Kondensator in Reihe geschaltet.
    Dabei hat der nach der Verbindung zum Oszilloskop eine bekannte Kapazität, wohingegen der der bei dem unbekannten Widerstand installierte eine zu bestimmende Kapazität besitzt.
    Der Aufbau ist in Abbildung \ref{fig:Kapazitiv} dargestellt.
    Den unbekannten Widerstand berechnet man mit Gleichung \ref{eqn:widerstand}.
    Die unbekannte Kapazität wird mit
    \begin{equation}
        C_x=C_2 \frac{R_4}{R_3}
        \label{eqn:kapazität}
    \end{equation}
    ausgerechnet.

    \subsection{Induktivitätsmessbrücke}
    Der Aufbau der Induktivitätsmessbrücke ist analog zu dem der Kapazitätsmessbrücke, außer dass die Kondensatoren durch Spulen ausgetauscht werden.
    Anschaulich kann man dies in Abbildung \ref{fig:Induktiv} erkennen.
    Damit gilt für die Berechnung des ohmschen Widerstandes wieder Gleichung \ref{eqn:widerstand} und die Induktivität der Spule, die bestimmt werden soll, wird mit 
    \begin{equation}
        L_x=L_2 \frac{R_3}{R_4}
        \label{eqn:induktivität}
    \end{equation}
    berechnet.

    \subsection{Maxwell-Brücke}
    Die Maxwell-Brücke wird ebenfalls zur Bestimmung der Induktivität einer Spule benutzt.
    Sie besitzt aber keine Spule $L_2$ in Reihe zu $R_2$, sondern stattdessen einen Kondensator, der parallel zu $R_4$ geschaltet ist.
    $R_4$ ist nun ein verstellbarer Widerstand. Das ist damit zu begründen, dass ein Kondensator mit geringen Verlusten einfacher zu realisieren ist, als eine Spule mit wenig Verlusten.
    Die Schaltskizze davon ist Abbildung \ref{fig:Maxwell}.
    Die Berechnung von $L_x$ erfolgt nun über
    \begin{equation}
        L_x=R_2 R_3 C_4
        \label{eqn:maxwell}
    \end{equation}
    und die von $R_x$ über Gleichung \ref{eqn:widerstand}.

    \subsection{Wien-Robinson-Brücke}
    Die Wien-Robinson-Brücke wird verwendet, um eine konstante Frequenz $\omega$ aus einem Frequenzspektrum rauszufiltern.
    Aufgebaut wird diese wie in Abbildung \ref{fig:Wien} und besitzt vier bekannte Widerstände, wobei zwei den Widerstand $R$, einer den Widerstand $R'$ und einer den doppelten Widerstand von $R'$ besitzt.
    Für den Betrag des Spannungverhätnis von der Brückenspannung und der Eingangsspannung gilt 
    \begin{equation}
       \biggl|\frac{U_{Br}}{U_S}\biggr|=\sqrt{\frac{1}{9}\frac{(\Omega^2-1)^2}{(1-\Omega^2)^2+9\Omega^2}}
       \label{eqn:filter}
    \end{equation}
    mit 
    \begin{gather}
        \Omega=\frac{\omega}{\omega_0}=\frac{f}{f_0}
        \label{eqn:Omega}
        \\
        \omega_0=\frac{1}{RC}
        \label{eqn:omega}
    \end{gather}

    \subsection{Klirrfaktor}
    Der Klirrfaktor stellt dar, wie fehlerfrei eine Sinusspannung ist. 
    Damit das das Überlagern der Spannung mit anderen Wellen gemeint.
    Der Klirrfaktor selbst wird durch das Verhältnis von der Sinus-Schwingung mit Überlagerungsschwingungen aufgestellt.
    Dieser kann mit der Wien-Robinson-Brücke gemessen werden.       
    Die Berechnung erfolgt mit 
    \begin{equation}
        k=\frac{1}{U_1}\sqrt{\sum_{i}^{n}{U_j^2}}
        \label{eqn:klirr1}
    \end{equation}\\
    mit
    \begin{equation}
        U_2=\frac{U_{Br}}{\sqrt{\frac{1}{9} \frac{(2^2-1)^2}{(1-2^2)^2+9*2^2}}}
        \label{eqn:oberwelle} 
    \end{equation}

\subsection{Fehlerrechnung}
Der Mittelwert einer Werteverteilung wird mit
\begin{equation}
    \bar{x}=\frac{1}{\symup{n}}\sum_{i=1}^{\symup{n}} x_i
    \label{eqn:MW}
\end{equation} 
bestimmt.

Den Fehler einer Größe berechnet man mit
\begin{equation}
    \increment\bar{x}=\sqrt{\frac{1}{\symup{n}\cdot(\symup{n}-1)}(\sum_{i=1}^{\symup{n}} (x_i - \bar{x}))}
    \label{eqn:fehler}
\end{equation}
Für Größen, die von mehreren Variablen, die jeweils einen Fehler besitzen, abhängen, berechnet man mit
\begin{equation}
    \increment f=\sqrt{\sum_{i=1}^{\symup{n}} (\frac{\partial f}{\partial x_i})^2 (\increment x_i)^2}
    \label{eqn:fehlerfortpflanzung}
\end{equation}


\section{Versuchsaufbau}
\label{sec:Versuchsaufbau}

Ziel des Versuches ist, die Trägheitsmomente verschiedener Körper zu bestimmen. 
Dafür wird der Körper auf einer Achse befestigt, die die Rotation des Körpers ermöglicht.
Diese Drillachse ist mit einer an einem festen Rahmen angebrachten Spiralfeder verbunden.
Auf der Halterung befindet sich eine Scheibe auf der sich der Auslenkwinkel ablesen lässt.
Der Körper kann nun zum Schwingen gebracht werden und die Schwingunsdauer T wird mit einer Stoppuhr gemessen.
Aus der Schwingunsdauer T, der Winkelrichtgröße D und dem Eigenträgheitsmoment der Drillachse $I_D$  
kann dann das Trägheismoment $I$ des Körpers bestimmt werden. 
\section{Durchführung}
\label{sec:Durchführung}
\subsection{Aufbau}
Auf einem Luftkissen, in einem Helmholtz-Spulenpaars, wurde eine Billiardkugel plaziert.
In dieser befindet sich ein Permanentmagnet.
Die Windungszahl des Helmholtz-Spulenpaares beträgt $N=195$ und es besitzt die Maße $d=\qty{0.138}{\meter}$ und $R=\qty{0.109}{\meter}$ wobei d der Abstand zwischen den Spulen und R der Radius ist.
Der Radius der Kugel beträgt $r_K=\qty{2.5}{\centi\meter}$ und die Masse $m_K=\qty{150}{\gram}$.
Das Luftkissen sorgt dafür, dass sich die Kugel hier annähernd reibungsfrei drehen kann.
An dem Aufbau befindet sich außerdem ein, auf die Kugel gerichtetes, Stroboskop mit einer Frequenz $f=\qty{5}{\hertz}$.
An der Kugel befindet sich ein kleiner Stab, auf dem ein weißer Punkt aufgemalt ist.



\subsection{Bestimmung des magnetischen Momentes über Präzession}
Hier wird die Kugel in Rotation versetzt.
Danach wird diese um einen kleinen Winkel ausgelenkt.
Mithilfe des Stroboskopes soll dann die Drehungsfrequenz der Kugel an die des Stroboskopes angepasst werden.
Dabei wird, nach dem Andrehen der Kugel, durch das Aufleuchten des weißen Punktes ermittelt, ob die Kugel sich schnell genug dreht.
Wenn der Punkt sich augenscheinlich mit der Drehrichtung der Kugel dreht, ist die Drehungsfrequenz größer als die des Stroboskopes.
Dann wird gewartet, bis der Punkt so erscheint, als würde er stehenbleiben, da dann die richtige Frequenz erreicht wurde.
Sofort wird das äußere Magnetfeld eingeschaltet und mithilfe einer Stoppuhr, die Umlaufdauer gemessen.
Die Messung wird jeweils drei mal für fünf verschiede Magnetfeldstärken durchgeführt. 

\subsection{Bestimmung des magnetischen Momentes über Gravitation}
In den Stab der Kugel wird eine Aluminiumstange gesteckt, an der sich ein verschiebbares Gewicht befindet.
Der Abstand des Schwerpunktes des Gewichts vom Permanentmagneten zu dem Mittelpunkt der Kugel ist $r$.
Hier werden 9 verschiedene Werte von $r$ eingestellt. 
Daraufhin wird für jedes $r$ die Magnetfeldstärke so angepasst, dass das durch das Magnetfeld erzeugte Drehmoment das der Gravitation genau kompensiert.
Dabei wird die Stromstärke des Stroms $I_H$, der durch das Helmholtz-Spulenpaar fließt, gemessen.

\subsection{Bestimmung des magnetischen Momentes über die Schwingungsdauer}
Bei der dritten Art der Bestimmung des magnetischen Momentes wird die Aluminiumstange wieder an der Kugel befestigt, aber diesmal ohne Gewicht.
Die Kugel wird nun innerhalb des Helmholtz-Spulenpaars, bei eingeschaltem Magnetfeld, um einen kleinen Winkel ausgelenkt.
Dadurch kann die daurauf Folgende Schwingung annähernd als harmonischer Oszillator beschrieben werden.
FÜr 9 verschiedene Stromstärken $I_H$ wird dann die Dauer von 10 Schwingungsperioden bestimmt.
\section{Auswertung}
\label{sec:Auswertung}
  \subsection{Winkelrichtgröße und Eigenträgheitsmoment}
  Die gemessene Kraft in Abhängigkeit vom dem Auslenkungswinkel wird in der Tabelle \ref{tab:tabelle1} dargestellt.
  Mit der Formel $\frac{r\cdot F}{\varphi}$ wird D berechnet.
  Mithilfe der gemessen Werte für $F$ und $\varphi$ kann mit $r = 20 \, \unit{\centi\meter}$ nun D bestimmt werden.
  Die einzelnen Werte von D sind ebenfalls in \ref{tab:tabelle1} eingetragen. 
  Mit $\overline{A} = \frac{1}{n} \sum_{i = 1}^{n} a_i$ wird nun das arithmetische Mittel gebildet. %% Brauch noch ne Quelle fürs Mittel
Einsetzen der Werte für D und $n = 11$ ergibt $\overline{D} = 0.02 \, \unit{\newton\meter}$.






\begin{figure}
  \centering
  \includegraphics{plot.pdf}
  \caption{Plot.}
  \label{fig:plot}
\end{figure}

\begin{table}
  \centering
  \caption{Tabelle 1}
  \label{tab:tabelle1}
 %% \sisetup{table-format=1.3, per-mode=reciprocal}
  \begin{tblr}{
     %% colspec = {S[table-format=3.0] S[table-format=2.1] S},
     colspec={S[table-format=2.0] S[table-format=2.0] S[table-format=2.0]},
      row{1} = {guard, mode=math},
    %%  vline{4} = {2}{-}{text=\clap{$\pm$}},
    }
    \toprule
    F \mathbin{/} \unit{\newton} & \varphi \mathbin{/} \unit{\degree} & \SetCell[c=2]{c} \symbf{D} \mathbin{/} \unit{\newton\meter} & \\
    \midrule
    0.016 &  20 & 0.009\\
    0.046 &  30 & 0.018\\
    0.066 &  40 & 0.019\\
    0.089 &  50 & 0.020\\ 
    0.11  &  60 & 0.021\\
    0.134 &  70 & 0.022\\
    0.162 &  80 & 0.023\\
    0.176 &  90 & 0.022\\
    0.18  & 100 & 0.021\\
    0.2   & 110 & 0.021\\
    0.23  & 120 & 0.022\\
    \bottomrule
  \end{tblr}
\end{table}

%% Siehe \autoref{fig:plot} und \autoref{tab:tabelle}!

\section{Diskussion}

Zur Berechnung der Abweichungen von den theoretischen Werten $w_t$ wurde die Gleichung 
\begin{equation}
    \frac{w_t-w_e}{w_t}\cdot \qty{100}{\percent}
    \label{eqn:ABweichung}
\end{equation}
\noindent verwendet, 
wobei $w_e$ der experimentell bestimmte Wert ist. Für den ersten Einzelspalt, mit dem Theoriewert $b_{1_t}\qty{0.075}{\milli\meter}$
und dem experimentellen Wert $b_1=\qty{0.0294(0.0020)}{\milli\meter}$, ergibt sich die Abweichung von $A_1\qty{60.8(2.7)}{\percent}$ vom Theoriewert. 
Für den zweiten Einzelspalt mit einem Theoriewert von $b_{1_t}\qty{0.15}{\milli\meter}$ und der experimentell bestimmten Spaltbreite $b_2=\qty{0.0901(0.0035)}{\milli\meter}$, folgt eine Abweichung von $\qty{39.9(2.3)}{\percent}$.
Die Spaltbreite des Doppelspaltes, gemessen $b_{DS}=\qty{0.059(0.008)}{\milli\meter}$ und der Abstand $s_{DS}=\qty{0.402(0.006)}{\milli\meter}$, 
besitzt nach Herstellerangaben die Werte $b_{DS_t}=\qty{0.1}{\milli\meter}$ und der Abstand $s_{DS}=\qty{0.4}{\milli\meter}$
Die Abweichung beträgt für die Spaltbreite $\qty{41(8)}{\percent}$ und der Spaltabstand eine von $\qty{0.5(1.5)}{\percent}$.
Die relativ großen Fehler können durch mögliche Fehler bei der Aufstellung der Ausgleichsfunktion, sowie dadurch erklärt werden, dass das Beugungsbild nur durch Augenmaß ausgerichtet
wurde und sich das größte Intensitätsmaximum dadurch möglicherweise nicht genau bei $x=\qty{25}{\centi\meter}$ befindet.

\label{sec:Diskussion}


%\begin{figure}
%  \centering
%  \includegraphics[width=\textwidth, angle=270]{Bilder/Stellung1.jpg}
%  \caption{Stellung 1}
%  \label{fig:Stellung1}
%\end{figure}
%\begin{figure}
%  \centering
%  \includegraphics[width=\textwidth, angle=270]{Bilder/Stellung2.jpg}
%  \caption{Stellung 2}
%  \label{fig:Stellung2}
%\end{figure}
%\begin{figure}
%  \centering
%  \includegraphics[width=\textwidth, angle=270]{Bilder/Data1.jpg}
%  \caption{Rohdaten Seite 1}
%  \label{fig:Data1}
%\end{figure}
%\begin{figure}
%  \centering
%  \includegraphics[width=\textwidth, angle=270]{Bilder/Data2.jpg}
%  \caption{Rohdaten Seite 2}
%  \label{fig:Data2}
%\end{figure}
%\begin{figure}
%  \centering
%  \includegraphics[width=\textwidth, angle=270]{Bilder/Data3.jpg}
%  \caption{Rohdaten Seite 3}
%  \label{fig:Data3}
%\end{figure}
%\begin{figure}
%  \centering
%  \includegraphics[width=\textwidth, angle=270]{Bilder/Data4.jpg}
%  \caption{Rohdaten Seite 4}
%  \label{fig:Data4}
%\end{figure}

\printbibliography{}

\end{document}
