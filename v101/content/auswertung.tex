\section{Auswertung}
\label{sec:Auswertung}
  \subsection{Winkelrichtgröße und Eigenträgheitsmoment}
  Die gemessene Kraft in Abhängigkeit vom dem Auslenkungswinkel wird in der Tabelle \ref{tab:tabelle1} dargestellt.
  Mit der Formel $\frac{r\cdot F}{\varphi}$ wird D berechnet.
  Mithilfe der gemessen Werte für $F$ und $\varphi$ kann mit $r = 20 \, \unit{\centi\meter}$ nun D bestimmt werden.
  Die einzelnen Werte von D sind ebenfalls in \ref{tab:tabelle1} eingetragen. 
  Mit $\overline{A} = \frac{1}{n} \sum_{i = 1}^{n} a_i$ wird nun das arithmetische Mittel gebildet. %% Brauch noch ne Quelle fürs Mittel
Einsetzen der Werte für D und $n = 11$ ergibt $\overline{D} = 0.02 \, \unit{\newton\meter}$.






\begin{figure}
  \centering
  \includegraphics{plot.pdf}
  \caption{Plot.}
  \label{fig:plot}
\end{figure}

\begin{table}
  \centering
  \caption{Tabelle 1}
  \label{tab:tabelle1}
 %% \sisetup{table-format=1.3, per-mode=reciprocal}
  \begin{tblr}{
     %% colspec = {S[table-format=3.0] S[table-format=2.1] S},
     colspec={S[table-format=2.0] S[table-format=2.0] S[table-format=2.0]},
      row{1} = {guard, mode=math},
    %%  vline{4} = {2}{-}{text=\clap{$\pm$}},
    }
    \toprule
    F \mathbin{/} \unit{\newton} & \varphi \mathbin{/} \unit{\degree} & \SetCell[c=2]{c} \symbf{D} \mathbin{/} \unit{\newton\meter} & \\
    \midrule
    0.016 &  20 & 0.009\\
    0.046 &  30 & 0.018\\
    0.066 &  40 & 0.019\\
    0.089 &  50 & 0.020\\ 
    0.11  &  60 & 0.021\\
    0.134 &  70 & 0.022\\
    0.162 &  80 & 0.023\\
    0.176 &  90 & 0.022\\
    0.18  & 100 & 0.021\\
    0.2   & 110 & 0.021\\
    0.23  & 120 & 0.022\\
    \bottomrule
  \end{tblr}
\end{table}

%% Siehe \autoref{fig:plot} und \autoref{tab:tabelle}!
