\section{Theorie}
\label{sec:Theorie}
Das Trägheitsmoment ist eine Tensorgröße, die starren Körpern zugeordnet werden kann und beschreibt, wie ein Körper sich bei
Rotationsbewegungen verhält.
Dabei setzt er die Winkelgeschwindigkeit eines Objektes mit dem resultierenden Drehimpuls in Beziehung mit
\begin{equation}
    \vec(L)=I\times\vec(\omega)\text{.}
    \label{eqn:Dreh1}
\end{equation}
Die Komponenten des Trägheitsmoments I sind dabei über die Gleichung
\begin{equation}
    I=\int r^2 \symup{d}m
    \label{eqn:Trägheit}
\end{equation}
bestimmt, für den Fall, dass die Drehachse durch den Schwerpunkt des Objekts verläuft. 
Falls dies nicht der Fall sein sollte, wird der steinersche Satz verwendet.
\begin{equation}
    I=I_{\symup{\omega}}+\symup{m} \symup{d}^2
    \label{eqn:Steiner}
\end{equation}   
m ist die Masse des Objekts und d der Abstand zur Drehachse. 

Für einen Schwingenden Körper kann mit ihm auch die Schwingungsdauer bestimmt werden über
\begin{equation}
    T=2 \symup{\pi} \sqrt(\frac{I}{\symup{D}})\text{.}
    \label{eqn:Schwingungsdauer}
\end{equation}
Dabei beschreibt D die sogenannte Winkelrichtgröße. Sie ist eine Konstante, die die Probortionalität zwischen dem 
Auslenkungswinkel des Objekts und dem rückstellenden Drehmoment wiedergibt.
\begin{equation}
    M=\symup{D} \varphi %aus der theoretischen Vorbereitung für den Versuch
    \label{eqn:Drehmom1}
\end{equation} 
Hierbei beschreibt M jetzt den Betrag des Drehmoments und $\varphi$ den Auslenkungswinkel.
Alternativ ist das Drehmoment außerdem mit
\begin{equation}
    \begin{gather}
        \vec(M)=\vec(r) \times \vec(F) \\ \vec(M)=\dot(\vec(L))
    \end{gather}
    \label{eqn:Drehmom2}
\end{equation}
beschreibbar.

Die Trägheitsmomente von Zylindern sehen dann entsprechend Gleichung \ref{eqn:Trägheit} für unterschiedliche
Drehachsen so aus.
\begin{equation}
    \begin{gather}
        I=frac{\symup{m}\symup{R}^2}{2} \text{Wenn die Rotationsachse senkrecht auf den Kreisoberflächen steht.}\\
        I=\symup{m} (frac{symup{R}^2}{4}+frac{\symup{h}^2}{12}) \text{Wenn die Rotationsachse senkrecht auf der Mantelfläche steht.}
    \end{gather}
\end{equation}
\cite{sample}
