\section{Diskussion}
\label{sec:Diskussion}
Dass die die Trägheismomente für die Figuren und Puppe negativ sind ist physkalisch nicht möglich.
Dies war bei den gemessenen und berechneten Werten jedoch der Fall.
Da dies auch bei den berechneten Werten auffällt, ist davon asuzugehen, dass der Fehler in der Berechnung des Eigenträgheitsmomentes der Drillachse liegt.
$I_D$ war nämlich der einzig negative Term in der Berechnung.
Außerdem war dieser der einzige Wert, der in beiden Methoden zur Bestimmung des Trägheitsmomentes verwentet wurde.
Im folgenden werden daher die Werte ohne Einbeziehen des Eigenträgheitsmomentes betrachtet.
Verglichen mit dem theoretisch bestimmten Wert weicht der experimentell bestimmte um eine ganze Größenordnung ab.
Dies kann zwar auf Messfehler hinweisen, jedoch sollte die grobe Näherung der Puppe, mit den Körperteilen als Zylinder, eine relativ große Fehlerquelle darstellen.
