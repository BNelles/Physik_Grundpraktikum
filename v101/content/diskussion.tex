\section{Diskussion}
\label{sec:Diskussion}
    \subsection{negative Trägheitsmomente}
    Dass die die Trägheismomente für die Figuren und Puppe negativ sind ist physkalisch nicht möglich.
    Dies war bei den gemessenen und berechneten Werten jedoch der Fall.
    Da dies auch bei den berechneten Werten auffällt, ist davon asuzugehen, dass der Fehler in der Berechnung des Eigenträgheitsmomentes der Drillachse liegt.
    $I_D$ war nämlich der einzig negative Term in der Berechnung.
    Außerdem war dieser der einzige Wert, der in beiden Methoden zur Bestimmung des Trägheitsmomentes verwentet wurde.
    Das Eigenträgheitsmoment wurde unter der Annahme, dass das Trägheitsmoment, des zur Drehachse senkrechten Eisenstabes, vernachlässigbar sei, berechnet.
    Dies ist eine mögliche Quelle des Fehlers, neben möglichen Fehlern bei der Messung und Fehlern bei der Regression.\\ 

    \subsection{Vergleich mit berechneten Werten}
    Im folgenden werden daher die Werte ohne Einbeziehen des Eigenträgheitsmomentes betrachtet.
    In Stellung 1 entspricht der experimentell bestimmte Wert im Vergleich zum berechneten $\frac{I_1}{I_{ges1}}=\qty{73.25}{\percent}$.\;
    In Stellung 2 ist das Verhältnis $\frac{I_2}{I_{ges2}}=\qty{52.73}{\percent}$.\\
    Verglichen mit dem theoretisch bestimmten Wert weicht der experimentell bestimte also in Stellung 1 um $\qty{26.75}{\percent}$ und in Stellung 2 um $\qty{47.27}{\percent}$ ab.
    Dies kann zwar auf Messfehler hinweisen, jedoch sollte die grobe Näherung der Puppe, mit den Körperteilen als Zylinder, eine relativ große Fehlerquelle bei dem theoretisch bestimmten Wert darstellen.
    Zudem kann hier auch das Vernachlässigen des Trägheismomentes der Drillachse zu einem Fehler führen.