\section{Versuchsaufbau}
\label{sec:Versuchsaufbau}
Ziel des Versuches ist, die Trägheitsmomente verschiedener Körper zu bestimmen. 
Dafür wird der Körper auf einer Achse befestigt, die aber Rotation des Körpers ermöglicht. %Der Nebensatz ist komisch. Ich glaube man kann den weglassen
Diese Drillachse ist mit einer an einem festen Rahmen angebrachten Spiralfeder verbunden.
Der Körper kann nun zum Schwingen gebracht werden und die Schwingunsdauer T wird mit einer Stoppuhr gemessen.
Aus der Schwingunsdauer T, der Winkelrichtgröße $\symbf{D}$ %würde ich ohne Mathemodus machen, oder? Ist ja eine Konstante
 und des Eigenträgheitsmoments der Drillachse $\symbf{I_D}$  
kann dann das Trägheismoment $\symbf{I}$ des Körpers bestimmt werden. %Das Trägheitsmoment ist doch das Eigenträgheitsmoment, oder meinst du Steiner
%Die Winkelscheibe und die Objekte müssten wir noch erwähnen denke ich