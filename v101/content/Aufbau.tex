\section{Versuchsaufbau}
\label{sec:Versuchsaufbau}

Ziel des Versuches ist, die Trägheitsmomente verschiedener Körper zu bestimmen. 
Dafür wird der Körper auf einer Achse befestigt, die die Rotation des Körpers ermöglicht.
Diese Drillachse ist mit einer an einem festen Rahmen angebrachten Spiralfeder verbunden.
Auf der Halterung befindet sich eine Scheibe auf der sich der Auslenkwinkel ablesen lässt.
Der Körper kann nun zum Schwingen gebracht werden und die Schwingunsdauer T wird mit einer Stoppuhr gemessen.
Aus der Schwingunsdauer T, der Winkelrichtgröße D und dem Eigenträgheitsmoment der Drillachse $I_D$  
kann dann das Trägheismoment $I$ des Körpers bestimmt werden. 