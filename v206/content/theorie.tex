\section{Theorie \cite{V206}}
\label{sec:Theorie}

Der Transport von Wärmeenergie verläuft in einem geschlossenen System von wärmeren in kältere Regionen, es ist jedoch möglich
durch das hinzufügen von mechanischer Energie von Außen die Richtung des Transports umzukehren. Dies wird mit einer 
Wärmepumpe realisiert, die über ein Transportmedium Wärmeenergie $Q_2$ aus dem kalten Reservoir Wärmeenergie $Q_1$ in das 
warme Reservoir. Beim Hinzufügen mechanischer Energie $A$ in einer Wärmepumpe muss mit Energieerhaltung das Verhältinis
\begin{equation}
    Q_1=Q_2+A
    \label{eqn:energie}
\end{equation}
gelten. Dass in Wärmepumpen dieses Verhältnis gilt, bringt den Vorteil, dass die aufgewendete Arbeit je nach Aufbau zunächst beliebig
klein sein darf, solange diese größer als 0 ist.
Desweiteren lässt sich im Allgemeinen mit dem zweiten Hauptsatz der Thermodynamik feststellen, dass
\begin{equation}
    \frac{Q_1}{T_1}-\frac{Q_2}{T_2}>0
\end{equation}
ist, was sich im Falle eines reversiblen Prozesses vereinfacht zu
\begin{equation}
    \frac{Q_1}{T_1}-\frac{Q_2}{T_2}=0 \text{ .}
\end{equation}
Zusammen mit der Gleichung \ref{eqn:energie} ergibt sich für einen reversiblen Prozess die Formulierung
\begin{equation}
    Q_1=A+\frac{T_2}{T1}Q_1
    \label{eqn:Q1}
\end{equation}
für $Q_1$.\\
\noindent Eine weitere Größe für eine Wärmepumpe ist die Gütezahl $\nu$, die über
\begin{equation}
    \nu=\frac{Q_1}{A}
    \label{eqn:güte}
\end{equation}
definiert ist. In Gleichung \ref{eqn:Q1} eingesetzt ergibt sich dann eine idealisierte Gütezahl $\nu_{\mathrm{ideal}}$, die mit
\begin{equation}
    \nu_{\mathrm{ideal}}=\frac{T_1}{T_1-T_2}
    \label{eqn:güteideal}
\end{equation}
berechnet wird. Für den realen Wert für $\nu$ gilt dann
\begin{equation}
    \nu_{\mathrm{real}}<\frac{T_1}{T_1-T_2} \text{ .}
\end{equation}
Für die aufzuwendende Arbeit, um Wärmeenergie in das kalte Reservoir zu bringen, wird also größer, je größer die Temparaturdifferenz
zwischen beiden Reservois ist. \\

\noindent Für einen zeitlich gemittelten Prozess lässt sich Gleichung \ref{eqn:güte} auch umschreiben als
\begin{equation}
    \nu=\frac{\Delta Q_1}{\Delta tN} \text{ ,}
    \label{eqn:gütemittel}
\end{equation}
wobei N die zeitlich gemittelte Leistung ist. Die Wärmemenge $\frac{\Delta Q_i}{\Delta t}$ kann dann auch zu
\begin{equation}
    \frac{\Delta Q_{\mathrm{i}}}{\Delta t}=(m_{\mathrm{i}}c_{\mathrm{w}}+m_{\mathrm{k}}c_{\mathrm{k}})\frac{\Delta T_i}{\Delta t}
    \label{eqn:wärmemenge}
\end{equation}
umformuliert werden. Diese Beziehung gilt für beide Behälter, wobei $m_{\mathrm{i}}c_{\mathrm{w}}$ die Wärmekapazität des Wassers
in den Behältern beschreibt und $m_{\mathrm{k}}c_{\mathrm{k}}$ die Wärmekapazität der Kupferbehälter.\\

\noindent Während des Transportprozesses kann der Massendurchsatz, das Gases, welches die
Wärme transportiert, bestimmt werden über
\begin{equation}
    \frac{\Delta Q_2}{\Delta t}=L\frac{\Delta m}{\Delta t} \text{ .}
\end{equation}
Hierbei bezeichnet $L$ die Verdampfungswärme und wird hier vereinfacht über die die Gleichung
\begin{equation}
    \mathrm{ln} \left(\frac{p}{p_0}\right)=-\frac{L}{R}\frac{1}{T}   \text{ ,}
    \label{eqn:verdampf}
\end{equation}
mit $R$ als allgemeiner Gaskonstante, bestimmt. Eingesetzt in \ref{eqn:wärmemenge}, wobei hier $T_2$ verwendet wird, 
entsteht die Gleichung
\begin{equation}
    \frac{\symup{d}m}{\symup{d}t}=\frac{m_2c_{\mathrm{w}}+m_{\mathrm{k}}c_{\mathrm{k}}}{L}\frac{\symup{d}T_2}{\symup{d}t} \text{ .}
    \label{eqn:masse}
\end{equation}
\\

\noindent Beim komprimieren eines Gases lässt sich die verrichtete Arbeit als das Integral
\begin{equation}
    A=-\int_{V_{\mathrm{a}}}^{V_{\mathrm{b}}}p \symup{d}V
\end{equation}
darstellen. Unter der Annahme, dass der Prozess adiabatisch ist, wird der Ausdruck
\begin{equation}
    N=\frac{1}{\kappa -1}\left(p_{\mathrm{b}}\left(\frac{p_{\mathrm{a}}}{p_{\mathrm{b}}}\right)^{\frac{1}{\kappa}}-p_{\mathrm{a}}\right)\frac{1}{\rho}\frac{\Delta m}{\Delta t}
    \label{eqn:leistung}
\end{equation}