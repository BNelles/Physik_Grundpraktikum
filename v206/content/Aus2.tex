\subsection{Berechnung des Massendurchsatzes}

Zur Bestimmung des Massendurchsatzes wird die Gleichung \ref{eqn:masse} verwendet. Die Summe der Wärmekapazitäten lässt sich hier 
direkt als $m_2c_{\symup{w}}+m_{\symup{k}}c_{\symup{k}}=\qty(13150)(\joule\per\kelvin)$ bestimmen. Dazu muss noch die Verdampfungswärme
$L$ bestimmt werden.

\begin{figure}
    \centering
    \includegraphics{verdampfplot.pdf}
    \caption{Der Druck bei dem kalten Reservoir gegen dessen Temparatur.}
    \label{fig:druckkalt}
\end{figure}

Mithilfe einer linearen Regression, die aussieht wie $ln(\frac{p}{p_0})=a\frac{1}{T_2}+b$ und der Gleichung \ref{eqn:verdampf},
lässt sich hier feststellen, dass die Verdampfungswärme $L=-a \cdot R$ einen Wert von $\qty{243(14)}{\joule\per\mole}$ annimmt.