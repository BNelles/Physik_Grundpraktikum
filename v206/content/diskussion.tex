\section{Diskussion}
\label{sec:Diskussion}

\subsection{Güteziffer}
Bei den Güteziffern sind die Abweichungen der Idealisierten von der Reellen 
\begin{gather*}
    91.36    \pm   0.18\\
    84.79    \pm   0.28\\
    82.00    \pm   0.50\\
    79.50    \pm   0.70\\
\end{gather*}
.
Hier ist die kleinste Abweichung $79.50    \pm   0.70$ mit $\nu_{\symup{real}}=1.81\pm   0.06 $   und     $\nu_{\symup{ideal}}=8.835    \pm   0.033$
und die größte $91.36    \pm   0.18$ mit $\nu_{\symup{real}}=2.72   \pm   0.04  $ und $   \nu_{\symup{ideal}}=31.500    \pm  0.500 $. 
Die durchgängig großen Abweichungen sind damit zu begründen, dass bei der idealisierten Wärmepumpe die Annahme getroffen wird, dass der Vorgang reversibel sei.
Dies ist nir der realen Wärmepumpe nicht vereinbar.
Bei dem Versuch ist diese Bediungung auch nicht näherungsweise gegeben, da die Abdichtund der Apparatur relativ schlecht sein sollte.
Zudem sind die Deckel auf die Wasserbehältern ohne eine Art Dichtung auf diese draufgelegt worden. 
Demnach ist damit zu rechnen, dass die Temparatur des Raumes die der beiden Gefäße beeinflusst.

\subsection{Massendurchsatz und Kompressorleistung}

Anhand der Tabelle \ref{tab:tabelle4} wird beobachtet, dass der Massendurchsatz sinkt von $\qty{78(5)}{\joule\kilo\gram}$ auf
$\qty{51.7(3.4)}{\joule\kilo\gram}$. Über die Zeit hinweg wird also weniger Masse zwischen den Reservoiren transportiert.
Im gleichen Zeitraum erhöht sich die Kompressorleistung jedoch von $\qty{7.3(0.5)}{\joule\per\second}$ auf $\qty{12.9(0.9)}{\joule\per\second}$, wie in 
\ref{tab:tabelle5} notiert ist. Somit lässt sich schließen, dass der Wärmetransport immer ineffizienter wird, da sich die
benötigte Leistung erhöht, die tatsächlich transportierte Masse aber kleiner wird. Auch nennenswert ist eine große Diskrepanz
zwischen der auf das Gas verrichteten Leistung und der tatsächlich zugeführten Leistung, was auf einen geringen 
Wirkungsgrad des Systems schließen lässt.
Auch dies könnte eine Folge der möglichen schlechten Abdichtungen sein.


