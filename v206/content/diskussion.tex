\section{Diskussion}
\label{sec:Diskussion}

\subsection{Güteziffer}
Bei den Güteziffern sind die Abweichungen der Idealisierten von der Reellen 
\begin{gather*}
    91.36    \pm   0.18\\
    84.79    \pm   0.28\\
    82.00    \pm   0.50\\
    79.50    \pm   0.70\\
\end{gather*}
.
Hier ist die kleinste Abweichung $79.50    \pm   0.70$ mit $\nu_{\symup{real}}=1.81\pm   0.06 $   und     $\nu_{\symup{ideal}}=8.835    \pm   0.033$
und die größte $91.36    \pm   0.18$ mit $\nu_{\symup{real}}=2.72   \pm   0.04  $ und $   \nu_{\symup{ideal}}=31.500    \pm  0.500 $. 
Die durchgängig großen Abweichungen sind damit zu begründen, dass bei der idealisierten Wärmepumpe die Annahme getroffen wird, dass der Vorgang reversibel sei.
Dies ist nir der realen Wärmepumpe nicht vereinbar.
Bei dem Versuch ist diese Bediungung auch nicht näherungsweise gegeben, da die ABdichtund der Apparatur relativ schlecht sein sollte.
Zudem sind die Deckel auf die Wasserbehältern ohne eine Art Dichtung auf diese draufgelegt worden. 
Demnach ist damit zu rechnen, dass die Temparatur des Raumes die der beiden Gefäße beeinflusst.

