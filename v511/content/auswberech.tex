\subsection{Berechnung verschiedener Parameter}

Die Werte für $n$ werden hier mit der Gleichung \ref{eqn:ladungsdichte} berechnet, die für $\bar{\tau}$ mit Gleichung \ref{eqn:mittlere_Flugdauer},
die mittleren Driftgeschwindigkeiten $v_d$ mit \ref{eqn:drift}, die Beweglichkeit $\mu$ mit \ref{eqn:Beweglichkeit}, die
Totalgeschwindigkeit $v$ mit \ref{eqn:total}, $\bar{l}$ mit \ref{eqn:Weglänge} und die Ladungsträger pro Atom $z$ mit 
Gleichung \ref{ladung_pro_atom}. Die Einträge für Silber stehen dabei in der Tabelle \ref{tab:paramsS}, die für Kupfer in 
der Tabelle \ref{eqn:paramsK} und die von Zink in Tabelle \ref{tab:paramsZ}.

Im Gegensatz zu den anderen beiden Metallen, hat Zink eine positive Ladungsträgerdichte im Leiter, worraus sich schließen lässt,
dass dort die psitiv geladenen Löcher als Leitendes Objekt überwiegen und in Silber und Kupfer es die negativ geladenen Elektronen sind.

Bei den Messungen für Kupfer und Zink wurden ein paar Messwerte vernachlässigt, da diese erst im Nachhinein entstanden sind
und durch die erhitzung der Spulen und die Magnetisierung der Magneten die Magnetfeldstärken stark verzerrt wurden. 
Diese hätten dann die Messung verfälscht.