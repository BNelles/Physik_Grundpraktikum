\section{Diskussion}


\label{sec:Diskussion}
\begin{equation}
    A_12=\Bigl|\fac{W_1-W_2}{W-2}\Bigr|
    \label{eqn:Abweichung}
\end{equation}

Die Abweichung von $n_s=\qty{-1.11(0.04)e28}{\coulomb\per\cubic\meter}$ und $n_{s_2}=\qty{-5.189(0.031)e+26}{\coulomb\per\cubic\meter}$ wird mit der Formel \ref{eqn:Abweichung} berechnet.
Daraus ergibt sich eine Abweichung von $A_{12}=\qty{2040+/-80}{\percent}$. 

Im Gegensatz zu den anderen beiden Metallen, hat Zink eine positive Ladungsträgerdichte im Leiter, worraus sich schließen lässt,
dass dort die psitiv geladenen Löcher als Leitendes Objekt überwiegen und in Silber und Kupfer es die negativ geladenen Elektronen sind.

Bei den Messungen für Kupfer und Zink wurden ein paar Messwerte vernachlässigt, da diese erst im Nachhinein entstanden sind
und durch die erhitzung der Spulen und die Magnetisierung der Magneten die Magnetfeldstärken stark verzerrt wurden. 
Diese hätten dann die Messung verfälscht.